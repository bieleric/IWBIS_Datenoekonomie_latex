\section{Ergebnisse}
Die in der Fallstudie betrachteten Beispiele zeigen, dass es verschiedene Möglichkeiten für Individuen gibt, ihre persönlichen Daten zu monetarisieren. 

\subsection{Datenerhebung}

\subsection{Kontrolle und Transparenz}

\subsection{Gegenwert für Individuen}

- Datenquellen
- Kontrolle: Widerruf, Transparenz, etc
- Gegenwert: Direkt Geld bei Plattformen, Preisnachlass bei Kaufland Card, Payback = Gegenstände, Produkte, Gutscheine

\begin{table}[!ht]
\begin{tabular}[h]{ |l|c|c|c|c|c| }
    \hline
    \thead{} & \thead{Cashback} & \thead{Payback} & \thead{BitsAboutMe} & \thead{Datum} & \thead{Invisibly} \\
    \hline
    \makecell{\textbf{Transparenz\textsuperscript{1}}} & & & & & \\
    \makecell{Konsument} & & & \cmark & \cmark & \\
    \makecell{Zeitraum} & & & \cmark & \cmark & \\
    \makecell{Verwendungszweck} & & & \cmark & \cmark & \\
    \makecell{Umfang} & & & \cmark & \cmark & \\
    \hline
    \makecell{\textbf{aktive Teilnahme\textsuperscript{2}}} & & & & & \\
    \makecell{bewusste Freigabe} & & & \cmark & \cmark & \\
    \makecell{Rückruf möglich} & & & \cmark & \cmark & \\
    \hline
    \makecell{\textbf{Gegenwert\textsuperscript{3}}} & & & & & \\
    \makecell{Geld} & & & \cmark & \cmark & \\
    \makecell{Preisnachlass} & & & \xmark & \xmark & \\
    \makecell{Prämien} & & & \xmark & \xmark & \\
    \hline
    \makecell{\textbf{Datenerhebung\textsuperscript{4}}} & & & & & \\
    \makecell{First-Party} & & & \xmark & \xmark & \\
    \makecell{Second-Party} & & & \xmark & \xmark & \\
    \makecell{Third-Party} & & & \cmark & \cmark & \\
    \hline
    \makecell{\textbf{Geschäftsmodell\textsuperscript{5}}} & & & & & \\
    \makecell{Datennutzer} & & & \xmark & \xmark & \\
    \makecell{Datenlieferanten} & & & \xmark & \xmark & \\
    \makecell{Datenvermittler} & & & \cmark & \cmark & \\
\end{tabular}
\caption{\label{tab:Auswertung der Fallstudie} Übersicht zur Auswertung der Fallstudie.}
\end{table}

\noindent \textsuperscript{1}Der Datenanbieter hat Einsicht über Art und Umfang der verarbeiteten Daten - unterteilt in nachfolgende Kategorien \newline

\noindent \textsuperscript{2}Der Datenanbieter nimmt aktiv an der Monetariserung seiner Daten teil - unterteilt in nachfolgende Kategorien \newline

\noindent \textsuperscript{3}Der Gegenwert, den der Datenanbieter für den Verkauf seiner Daten erhält - unterteilt in nachfolgende Kategorien\newline

\noindent \textsuperscript{4}Die Art der Datenerhebung aus Sicht des Datenkonsumenten gemäß Kapitel \ref{datenoekonomie} Datenökonomie - unterteilt in nachfolgende Kategorien\newline

\noindent \textsuperscript{5}Das Geschäftsmodell der Plattform oder des Unternehmens gemäß Kapitel \ref{datenoekonomie} Datenökonomie - unterteilt in nachfolgende Kategorien\newline