\section{Datenschutz}
\subsection{Personenbezogene Daten}
Laut der Definition in der Datenschutz Grundverordnung (\gls{DSGVO}) im Kapitel 1 Artikel 4 Nummer 1 werden personenbezogene Daten als diejenigen Informationen bezeichnet, mit denen sich \gls{natuerlichePersonG}en identifizieren lassen. Die Identifikation kann auf direktem oder indirektem Wege erfolgen, insbesondere durch die Zuordnung eines Namen zu Kennnummern, Standortdaten oder anderen psychischen, physiologischen, genetischen, wirtschaftlichen, kulturellen oder sozialen Merkmalen der natürlichen Person. \cite{DSGVO_Art4} Gemäß der europäischen Union werden Teilinformationen, die zusammen zur Identifizierung einer natürlichen Person dienen, ebenfalls als personenbezogene Daten kategorisiert. Werden diese Daten anonymisiert und lassen keinen Schluss auf die natürliche Person zu, so werden diese Daten nicht mehr als personenbezogen betrachtet. Beispiele für personenbezogene Daten sind Name, Vorname, Privatanschrift, E-Mail-Adresse mit Namen, Standortdaten und IP-Adresse. Als nicht personenbezogene Daten werden beispielsweise Handelsregisternummer, anonymisierte E-Mail-Adressen und generell anonymisierte Daten betrachtet. \cite{PersBezDaten_2021}

\subsection{Besonders schützenswerte personenbezogene Daten} \label{DSGVO_besonders}
Zu erwähnen ist, dass es besonders schützenswerte Daten einer natürlichen Person gibt. Die Verarbeitung dieser Daten ist entsprechend der DSGVO grundsätzlich untersagt. Zu diesen Daten gehören mitunter die rassische und ethnische Herkunft, politische Meinungen, religiöse oder weltanschauliche Überzeugungen, die Gewerkschaftszugehörigkeit, sowie genetische und biometrische Daten, Gesundheitsdaten oder Daten zum Sexualleben oder der sexuellen Orientierung einer natürlichen Person. Die im Artikel 9 Absatz 2 beschriebenen Fälle bilden die Ausnahme zur Verarbeitung dieser Daten. \cite{DSGVO_Art9}