\documentclass[
 openright,
 %twoside, % beidseitig für Druck
 a4paper
]{scrreprt}
%scrreprt
\input{htwcd/htwcd_content.sty}

% Vorgaben zum Seitenrand	
\geometry{
	top=30mm, 
	bottom=30mm, 
	%headsep=15mm, 
	inner=40mm, 
	outer=20mm, 
	%left=40mm,
	%right=20mm,
	%footskip=15mm,
}

\usepackage{scrlayer-scrpage}
\pagestyle{scrheadings}
\clearpairofpagestyles
\ofoot{\pagemark}
\raggedbottom

% Codierung
\usepackage[utf8]{inputenc}
\usepackage[T1]{fontenc}
\usepackage[ngerman]{babel}
\usepackage{csquotes}	% Anführungszeichen
\usepackage{amssymb}
\usepackage{pifont}

\newcommand{\cmark}{\ding{51}}
\newcommand{\xmark}{\ding{55}}

% Grafiken
\usepackage{graphicx}
\graphicspath{ {./images/} }
\usepackage{float}
\usepackage{longtable}
\usepackage{placeins}

% Tabellen
\usepackage{makecell}

\renewcommand\theadalign{bc}
\renewcommand\theadfont{\bfseries}
\renewcommand\theadgape{\Gape[4pt]}
\renewcommand\cellgape{\Gape[4pt]}

% Quelltext
\usepackage{listings}
\usepackage{scrhack}

% Links
\usepackage{xurl}
\usepackage[bookmarks,%
bookmarksopen=false,% Klappt die Bookmarks in Acrobat aus
colorlinks=true,%
linkcolor=black,%
citecolor=black,%
urlcolor=black,%
]{hyperref}
\usepackage{titleref}

\usepackage{datetime}


\usepackage{tikz}
\usetikzlibrary{positioning,shadings}
\usetikzlibrary{arrows}

% ======================================================
% Informationen für das Dokument und Titelseite:
% ======================================================
\faculty{Fakultät Informatik/Mathematik}

\title{Persönliche Daten in der Datenökonomie}

\author{
	Maria Mukian\\
	Philipp Steigler\\
	Eric Hans Gero Biele
}

\professor{Prof. Dr. Jürgen Anke}

\newdate{abgabe}{26}{01}{2021}
\date{\displaydate{abgabe}}
% ======================================================


% ======================================================
% Literatur/Quellen und Akronyme/Glossar:
% ======================================================
\usepackage[
backend=bibtex,
sortlocale=de_DE, 
bibencoding=utf8,
style=numeric, 
citestyle=numeric,
doi=false,
isbn=false,
url=false,
sorting=nty
]{biblatex}	% BibTeX
\bibliography{Bibliography}

\usepackage[acronym]{glossaries}
\makeglossaries
% ------------------------------------------
% Glossar  
% keine Umlaute im entry verwenden   
\newglossaryentry{natuerlichePersonG}
{
name={natürliche Person},
description={\glqq Eine natürliche Person meint den Menschen als Rechtssubjekt und somit als Träger von Rechten und Pflichten.\grqq{} \cite{NatPerson_2018}}
}

%============================================
% Akronyme
%============================================
\newglossaryentry{DSGVO}{
	type=\acronymtype, 
	name={DSGVO}, 
	description={Datenschutz-Grundverordnung}
}
% ======================================================
\pagenumbering{roman}

\begin{document}
% ==============================================
% Deckblatt
% ==============================================
\maketitle

% ==============================================
% Abstract
% ==============================================
\setcounter{page}{1}
\pagenumbering{gobble}
\thispagestyle{empty}
\vspace*{5cm}
\begin{center}
\begin{minipage}{0.9\textwidth}
\chapter*{Abstract}

Die vorliegende Arbeit gibt einen Überblick über die Möglichkeiten der aktiven Monetarisierung von persönlichen Daten mit deren Potenzial und Einschränkungen. Im Zuge dessen wurden die Grundlagen der Datenökonomie, sowie der monetäre Wert von Daten untersucht und darauf aufbauend eine Fallstudie durchgeführt, welche die Basis dieser Arbeit bildet. Die Ergebnisse der Fallstudie führten zu einer Kategorisierung der Monetarisierungsmöglichkeiten durch den Verbraucher. Diese Arbeit ist sowohl für IT- und Wirtschaftswissenschaftler, als auch für Unternehmen und Verbraucher selbst interessant, die sich auf datengetriebene Technologien stützen.

\vspace*{1.5cm}
\end{minipage}
\end{center}
\clearpage

% ==============================================
% Verzeichnisse:
% ==============================================
% Inhaltsverzeichnis
\pagenumbering{roman}
\setcounter{page}{1}
\tableofcontents
% Abkürzungsverzeichnis
% ACHTUNG: Akronyme und Glossar muss (ähnlich wie bibtex) extra kompiliert werden über makeglossaries (in texmaker bspw. "makeglossaries %")
\printglossary[type=\acronymtype]
\addcontentsline{toc}{chapter}{Akronyme}
% Glossar
\printglossary[type=main]
\addcontentsline{toc}{chapter}{Glossar}
% Abbildungsverzeichnis
\listoffigures
\addcontentsline{toc}{chapter}{Abbildungsverzeichnis}
% Tabellenverzeichnis
\listoftables
\addcontentsline{toc}{chapter}{Tabellenverzeichnis}

% ==============================================
% Textteil
% Richtlinie: 25-40 Seiten
% \section{}, \subsection{}, \subsubsection{}
% ==============================================
\clearpage
\pagenumbering{arabic}
\chapter{Einleitung}
% * Einleitung
\section{Motivation}
Die Verbreitung des Internets auf Computern und mobilen Endgeräten hat zu einer Produktion riesiger Datenmengen geführt. Diese Daten -- insbesondere personenbezogene Daten -- haben sich zu einer wichtigen Ressource entwickelt, mit der Unternehmen einen Mehrwert erzeugen. Im Weiteren entstand daraus eine Datenökonomie, die allein auf dem Handel und der Verarbeitung personenbezogener Daten beruhen. \cite{humanDemand_2020} Dabei werden Daten meist jedoch nicht direkt von den Erzeugern, den individuellen Personen, verkauft. Stattdessen erheben große Internetdienste Anspruch auf die Daten ihrer Nutzer durch Klauseln in den Nutzungsbedingungen und verkaufen diese dann an Dritte weiter -- oftmals ohne Nutzer im Einzelfall darüber in Kenntnis zu setzen. Obwohl Individuen durch Erzeugung von Daten die Grundlage solcher Ökosysteme bilden, werden sie von Unternehmen bislang nur als Datenquelle gesehen und somit ausgenutzt. \cite{monetizingData_2016} \newline

\noindent Viele Personen werden sich in letzer Zeit allerdings zunehmend bewusst, dass Technologieunternehmen diese Daten sammeln und verkaufen. Darüber hinaus wird ihnen deshalb bewusst, dass ihre persönlichen Daten einen gewissen Wert haben und dass sie mehr Kontrolle über deren Verwendung haben sollten. Rantanen und Koskinen fanden in einer Studie heraus, dass Personen zwei wesentliche Forderungen an eine faire Datenökonomie stellen: transparente Kommunikation und eine aktive Rolle im Handel mit den eigenen Daten. Eine aktive Rolle in der Datenökonomie meint dabei, dass Individuen die Nutzung ihrer eigenen Daten kontrollieren und für die Teilnahme einen entsprechenden Gegenwert erhalten wollen. \cite{humanDemand_2020} \newline

\noindent Bereits im Jahr 2016 stellte Batineh et. al jedoch fest, dass es an Plattformen mangelt, auf denen Individuen als primäre Datenerzeuger ihre Daten für einen entsprechenden Gegenwert selbst anbieten können. \cite{monetizingData_2016}
% * Ziel und Methodik
\section{Ziel und Methodik}

Aus diesem Grund beschäftigen wir uns im Folgenden mit den verschiedenen Möglichkeiten für Personen, sich aktiv in der Datenökonomie zu beteiligen. In dieser Arbeit stellen wir dar, bei welchen Diensten individuelle Personen ihre Daten anderen Akteuren der Datenökonomie zur Verfügung stellen können und welchen Gegenwert sie dafür jeweils erhalten. Wir fokussieren uns dabei ausschließlich auf persönliche Daten, da diese sehr empfindliche Informationen über Individuen enthalten und deshalb für Unternehmen von großem Interesse sind. Zudem betrachten wir für jeden Fall, mit welcher Kontrolle und Transparenz die Daten geteilt werden. Unsere Forschungsfrage lautet deshalb:
\\
\\
\centerline{\textit{Welche Möglichkeiten gibt es für Personen, aktiv ihre Daten ökonomisch zu verwerten?}}
\\
\\
Um diese Frage zu beantworten haben wir ...

\chapter{Grundlagen}
% * Datenschutz
\section{Datenschutz}
\subsection{Personenbezogene Daten}
Laut der Definition in der Datenschutz Grundverordnung (\gls{DSGVO}) im Kapitel 1 Artikel 4 Nummer 1 werden personenbezogene Daten als diejenigen Informationen bezeichnet, mit denen sich \gls{natuerlichePersonG}en identifizieren lassen. Die Identifikation kann auf direktem oder indirektem Wege erfolgen, insbesondere durch die Zuordnung eines Namen zu Kennnummern, Standortdaten oder anderen psychischen, physiologischen, genetischen, wirtschaftlichen, kulturellen oder sozialen Merkmalen der natürlichen Person. \cite{DSGVO_Art4} Gemäß der europäischen Union werden Teilinformationen, die zusammen zur Identifizierung einer natürlichen Person dienen, ebenfalls als personenbezogene Daten kategorisiert. Werden diese Daten anonymisiert und lassen keinen Schluss auf die natürliche Person zu, so werden diese Daten nicht mehr als personenbezogen betrachtet. Beispiele für personenbezogene Daten sind Name, Vorname, Privatanschrift, E-Mail-Adresse mit Namen, Standortdaten sowie auch Informaionen zu persönlichen Interessen oder dem Konsumverhalten. Als nicht personenbezogene Daten werden beispielsweise Handelsregisternummer, anonymisierte E-Mail-Adressen und generell anonymisierte Daten betrachtet. \cite{PersBezDaten_2021}

\subsection{Besonders schützenswerte personenbezogene Daten} \label{DSGVO_besonders}
Zu erwähnen ist, dass es besonders schützenswerte Daten einer natürlichen Person gibt. Die Verarbeitung dieser Daten ist entsprechend der DSGVO grundsätzlich untersagt. Zu diesen Daten gehören mitunter die rassische und ethnische Herkunft, politische Meinungen, religiöse oder weltanschauliche Überzeugungen, die Gewerkschaftszugehörigkeit, sowie genetische und biometrische Daten, Gesundheitsdaten oder Daten zum Sexualleben oder der sexuellen Orientierung einer natürlichen Person. Die im Artikel 9 Absatz 2 beschriebenen Fälle bilden die Ausnahme zur Verarbeitung dieser Daten. \cite{DSGVO_Art9}
% * Datenökonomie
\section{Datenökonomie} \label{datenoekonomie}
Unter Datenökonomie (engl. data economy) versteht man die wirtschaftliche Nutzung von Daten. Die wirtschaftliche Nutzung personenbezogener Daten wird dabei zusätzlich rechtlich durch die DSGVO eingeschränkt. Aber auch nicht personenbezogene Daten werden ökonomisch verwertet, so werden beispielsweise Prozess- und Maschinendaten gesammelt, aufbereitet und ökonomisch verwertet. \cite{bpb_2019} Meist werden die Daten dabei aggregiert und anderweitig aufbereitet, sodass sie im Falle der Erfassung von personenbezogenen Daten nicht mehr auf einzelne Personen zurückzuführen sind und daher fortan nicht mehr als personenbezogen betrachtet werden und entsprechend verwertet werden dürfen. \newline

\noindent Häufig werden Daten dazu verwendet, um Recherche- und Transaktionskosten zu senken und anhand derer präzise Entscheidungen zu treffen. Daher entsteht keine direkte Monetarisierung aus den Daten, weshalb der Wert dieser nicht eindeutig bestimmbar ist, sondern meist nur ein Schätzwert ist. Neben dieser Eigenschaft weisen Daten noch weitere Eigenschaften auf, die in der Datenökonomie eine wesentliche Rolle spielen. Einerseits haben Daten keinen Wertverlust durch deren Nutzung, was sie unerschöpflich macht. Andererseits kann deren Wert aber auch genau durch die Irrelevanz dieser sinken. Einzelne Daten haben meist wenig Relevanz und haben demzufolge einen geringen Wert. Erst durch die Aggregation mit anderen Daten entsteht ein Wert dieser, wodurch folglich Prognosen, Kennzahlen und weitere Vorgehen abgeleitet werden können. Daraus lässt sich gemäß eines Berichts der \gls{UN} eine Wertschöpfungskette für Daten herleiten. Diese besteht aus der Datenerfassung, der Speicherung und Organisation der Daten, der Analyse dieser und der folgenden Berichterstellung, woraus demnach die oben erwähnten Entscheidungen resultieren. \cite{un_2019} \newline

\noindent Aufgrund des enormen Potentials sind datengetriebene Geschäftsmodelle enstanden, die im Folgenden beleuchtet werden. Jedoch muss dabei vorerst unterschieden werden, wie differenziert die Datenerhebung stattfinden kann. Die International Data Corporation hat in einer Studie, zusammen mit der Open Evidence, drei Arten der Datenerhebung für die Europäische Kommission ermittelt. Zum einen sind dies \textit{First-Party-Daten}, die von Unternehmen über ihre Kunden gesammelt werden, um interne Analysen durchzuführen oder sie an Dritte zu verkaufen. \textit{Second-Party-Daten} dagegen sind Daten die in Kooperation mit anderen Unternehmen durch beispielsweise Werbeaktionen gesammelt werden und sich in gemeinsamen Besitz befinden. Bei der dritten Form der Datenerhebung, der sogenannten \textit{Third-Party-Daten}, werden Daten durch \gls{webScrapingG} oder durch den Kauf bei Datenanbietern erfasst. Dies ist der erste Abschnitt, der oben beschriebenen Wertschöpfungskette und damit auch die Grundlage des datengetriebenen Geschäftsmodells. Wurden die Daten erfasst, lässt sich ein Geschäftsmodell daraus ableiten. Oft sind datengetriebene Geschäftsmodelle \textit{\glqq as-a-service\grqq{}-Modelle}, das heißt die Daten dienen dem Nutzer als Service. So können zum Beispiel Produktvorschläge, Rabatte oder Werbung passend zum Kaufverhalten des Nutzers gemacht werden, die mitunter bei \gls{SaaS}- und \gls{PaaS}-Modellen zum Einsatz kommen. Ein weiteres Geschäftsmodell kann die Datenanalyse selbst sein, also \textit{Analytics-as-a-Service}, wobei ein Mehrwert für den Nutzer allein durch die aufbereiteten Daten entsteht. Als Beispiel wären hierbei Fitnesstracker zu nennen, die die Gesundheitsdaten der Nutzer erfassen und für diesen verständlich darstellen. Darüberhinaus kann ein weiteres Geschäftsmodell die Beratung von Unternehmen im Digitalisierungsprozess, \textit{Consultancy-as-a-Service}, sein, um diese in der eigenen Datenerfassung und -analyse zu unterstützen, damit diese wiederum First-Party-Daten auswerten können und ihr Geschäftsmodell verbessern können. \newline

\noindent Daraus leiten die Autoren des oben genannten Berichts drei Klassen von datengetriebenen Geschäftsmodellen ab. Dies sind zum einen Datennutzer, die ihre ermittelten Daten für die Verbesserung des eigenen Geschäftsmodells nutzen. Zum anderen sind dies Datenlieferanten, die analysierte Daten zur weiteren Nutzung bereitstellen und schließlich Datenvermittler, welche durch Beratung und Bereitstellung von Infrastrukturen hinsichtlich der Datenerfassung ein Geschäftsmodell betreiben. \cite{smart_2013} \newline

\noindent Zusätzlich werden zwei weitere Begriffe eingeführt: der \textit{Datenkonsument} und der \textit{Datenanbieter}. Dabei handelt es sich um Synonyme für das Unternehmen beziehungsweise den Nutzer. Das Unternehmen nimmt in der Datenökonomie vorwiegend die Rolle des Datenkonsumenten ein. Das heißt, es möchte anhand der gewonnen Daten oben genannte Mehrwerte generieren. Der Nutzer dagegen nimmt die Rolle des Datenanbieters ein, da dieser dem Datenkonsumenten die relevanten Daten anbietet und liefert. 
% * Wert der Daten
\section{Der ökonomische Wert von personenbezogenen Daten} \label{oekonomischerWert}

Die Studie \textit{Ökonomischer Wert von Verbraucherdaten für Adress- und Datenhändler}, welche im Auftrag des Bundesministeriums der Justiz und für Verbraucherschutz durchgeführt wurde, ermittelte annäherungsweise den ökonomischen Wert von Nutzerdaten. Sie analysierten dafür die zehn umsatzstärksten Unternehmen Deutschlands im Bereich Datenhandel mit personenbezogenen Daten. Allein 2014 erwirtschafteten diese zehn Unternehmen einen Gesamtumsatz von etwa 450 Mio. Euro, wobei die Schufa beispielsweise im Besitz von 797 Mio. Einzeldaten zu 66,4 Mio. Personen ist. Das bedeutet im Umkehrschluss, dass die Schufa, bei einer Einwohnerzahl von 83,2 Mio. Menschen in Deutschland (Stand 2021 \cite{einwohnerzahl_2021}), Daten von knapp 80\% der deutschen Bevölkerung hat, wobei auf jeden Einzelnen durchschnittlich zwölf Einzeldaten entfallen. Die Studie kommt zu dem Ergebnis, dass der durchschnittliche Wert eines Datensatzes 0,86 Euro beträgt. Ein Datensatz wird dabei aus Post- und E-Mailadressen beziehungsweise Personendaten erstellt, die mit personenbezogenen Merkmalen, wie Konsumverhalten, Bonität, etc., angereichert sind. Bei einer Verwertbarkeit eines Datensatzes von 30 Jahren erzielten die untersuchten Unternehmen damit einen durchschnittlichen Umsatz von 26 Euro pro personenbezogenen Datensatz, wobei dieser je nach Unternehmen zwischen vier und 77 Euro schwankt. \cite{Wert_der_Daten_2017}

\chapter{Fallstudie}
% * Erklärung zur Auswahl der Fallbeispiele
In diesem Kapitel werden eine Reihe an Diensten betrachtet, bei denen individuelle Personen aktiv ihre persönlichen Daten ökonomisch verwerten können. Die Menge an Plattformen wurden hauptsächlich durch Websuchen mit folgenden Suchbegriffen zusammengetragen: \textit{personal data monetization}, \textit{monetize personal data}, \textit{data licensing}, \textit{Daten zu Geld machen}, \textit{persönliche Daten verkaufen}, \textit{Datenökonomie und welche Platteformes es gibt, um persönliche Daten zu verkaufen}. Aus diesem Pool an Suchbegriffen bzw. Wortgruppen wurden die nachfolgenden Dienste ausgewählt. Ausschlaggebend für Payback war allerdings der Bekanntheitsgrad.
% * Payback
\section{Payback} \label{Payback}
Payback ist ein Bonussystem, welches 2000 in Deutschland eingeführt wurde und seit 2018 auch in Italien, Indien, Polen, Mexiko und Österreich verfügbar ist. Außerdem gehört es seit 2010 zur American Express Gruppe. \cite{Payback_Info}

\subsection{Rollen und Ziele}
Payback ist ein Treuepunkteprogramm, mit dem Kunden bei jedem Einkauf prozentual Punkte sammeln und diese gegen Prämien und Rabatte einlösen. Der Anbieter hat hierfür Verträge mit über 30 stationären Partnern und 600 Online Shops. \cite{Payback} Demzufolge existieren drei Rollen im Payback-Netzwerk: \textit{Payback} als Anbieter, der \textit{Kunde} und die \textit{Partnerunternehmen}. \newline

\noindent \textbf{Payback:} Das Ziel von Payback besteht vor allem darin, als zentrale Instanz Daten über seine Nutzer zu sammeln und diese auszuwerten. Auf diese Art kann die Plattform das Einkaufsverhalten über sämtliche Händler hinweg analysieren und seinen Nutzern dahingehend personalisierte Werbung anzeigen. \newline

\noindent \textbf{Partnerunternehmen:} Die Partnerunternehmen schließen mit Payback zur Teilnahme am Bonusprogramm einen Vetrag. Damit erhoffen sich die Händler eine stärkere Kundenbindung, da Besitzer einer Payback-Karte sich daran orientieren, ob sie beim Einkauf Payback-Punkte sammeln können. Darüber hinaus können Partnerunternehmen mithilfe von Payback-Coupons Werbeaktionen unterstützen, um einerseits Aufmerksamkeit auf Warengruppen und Termine zu richten und andererseits neue Kunden zu gewinnen, die bereits eine Payback-Karte besitzen. Letztendlich erhalten die Partner Zugriff auf Analyseergebnisse von Payback zum allgemeinen Kundenverhalten -- diese beinhalten nicht nur eigene Kunden, sondern alle Kunden der Payback-Partner. \newline

\noindent \textbf{Kunde:} Personen beantragen kostenlos eine Payback-Karte bzw. registrieren sich in der Payback-App, um am Bonusprogramm teilzunehmen. Diese zeigt der Kunde beim Bezahlvorgang an der Kasse vor und erhält einen nach Unternehmen unterschiedlichen Bonus in Form von Punkten gutgeschrieben. Online ist das Punktessammeln im Buchungsprozess bei einigen Händlers integriert, bei anderen muss der Kunde seinen Einkauf direkt über die Payback-Webseite oder App starten. Gesammelte Punkte können später für Rabatte, Gutscheine und weitere Prämien eingelöst oder direkt ausgezahlt werden.

\subsection{Datenerhebung}
Um am Payback-Programm teilzunehmen, müssen Benutzer folgende Basisdaten an Payback als Anbieter übermitteln: Name, Geburtsdatum und Adresse. Dabei ist es wichtig zu erwähnen, dass eine Teilnahme auch ohne Angabe dieser Daten möglich ist -- jedoch können dann keine gesammelten Punkte eingelöst werden. \cite{Payback_Teilnahme} Darüber hinaus können Personen freiwillig weitere Basisdaten angeben. \cite{Payback_Datenschutz} \newline

\noindent Bei der Datenerhebung gibt es jedoch nach der Art der Registrierung wesentliche Unterschiede. Beantragt der Kunde seine physische Payback-Karte bei einem der stationären Partner, so erhält auch dieser Zugriff auf die Basisdaten. Beantragt der Kunde hingegen eine neutrale Payack-Karte bei Payback selbst oder registriert sich online zur Verwendung der App, so bleiben alle persönlichen Basisdaten bei Payback. Partnerunternehmen übermitteln in jedem Fall die gesammelten Einkaufsdaten an die Plattform. Dies umfasst Payback-Kundennummer, eine Auflistung aller Waren/Dienstleistungen mit Preisen sowie den Rabattbetrag, Ort und Zeitpunkt des Einkaufs. Eine Ausnahme bilden Apotheken -- diese erhalten keinen Zugriff auf Basisdaten von Payback und melden auch keine Einkaufsdaten an Payback, da es sich gemäß Kapitel \ref{DSGVO_besonders} um besonders schützenswerte Daten handelt. \cite{Payback_Datenschutz} \newline

\noindent Payback speichert die Daten der Kunden nur solange diese aktiv am Programm teilnehmen. Steuerrechtlich werden diese jedoch zehn Jahre lang aufbewahrt. Die Kommunikationsdaten, zum Beispiel bei Kontaktierung des Service Centers werden nach spätestens sechs Jahren gelöscht. Darüber hinaus weist Payback ausdrücklich darauf hin, dass keine Adressdaten mit Partnerunternehmen geteilt werden. \cite{Payback_Datenschutz} \newline

\noindent Weiterhin kann der Nutzer frei entscheiden, ob seine Daten für Zwecke der Marktforschung und Werbung verwendet werden dürfen. Wenn der Nutzer einwilligt, werden passende werbliche Angebote für ihn ausgewählt. Diese erfolgen durch die Auswertung der Daten zur Mustererkennung im Einkaufsverhalten. Die Angebote werden entweder postalisch, per E-Mail oder per SMS mitgeteilt. Diese Daten werden durch Payback und die Partnerunternehmen zum Zwecke der Werbung verarbeitet. Der Kunde hat dabei jederzeit die Möglichkeit die Einwilligung zu widerrufen. Sobald die Einwilligung zur Verarbeitung der Daten für Werbezwecke vorliegt, wird anhand dessen ebenfalls eine Werbeplanung und Erfolgskontrolle durchgeführt. \cite{Payback_Datenschutz} \newline

\subsection{Daten bei Payback monetarisieren}
Eine Monetarisierung von Daten findet bei Payback für Individuen indirekt statt. Benutzer teilen ihre persönlichen Daten mit der Plattform und erhalten Punkte, die sie später für verschiedenen Gegenwerte einlösen können. \newline

\noindent Die oben genannten Basisdaten müssen für die Teilnahme am Bonusprogramm zwingend mit Payback geteilt werden, obwohl Individuen dafür keine Punkte und somit keinen Gegenwert erhalten. Erst mit dem Teilen von Einkaufsdaten werden Personen mit Punkten vergütet. Der Betrag an Punkten varriert dabei von Unternehmen zu Unternehmen -- insgesamt werden aber Punkte im Wert von 0,5 bis 4\% der Kaufsumme gutgeschrieben. Häufig können Kunden mit Coupons oder Sonderaktionen jedoch einen vielfachen Betrag an Punkten sammeln. \newline

\noindent Gesammelte Puntkte können anschließend für verschiedene Gegenwerte eingelöst werden. Einerseits ist eine direkte Auszahlung als Geldbetrag möglich. Ein Payback-Punkt hat dabei den Wert von einem Cent (0,01€) und eine Auszahlung ist ab 200 Punkten -- also 2€ -- möglich. \cite{Payback_Teilnahme} Andererseits können Kunden ihre Punkte auch für nichtmonetäre Gegenwerte einlösen: Einkaufs- und Geschenkgutscheine für viele Partner, Tanken und Shoppen bei Aral, Flugmeilen bei Lufthansa (1 Payback-Punkt entspricht 1 Meile), Spenden in der Payback-Spendenwelt oder direkt im Payback-Prämienshop für Produkte und Gutscheine. \cite{Payback_Einlösen} Abschließend ist es noch wichtig zu erwähnen, dass die gesammelten Punkte eines jeden Jahres zum 30. September verfallen, wenn diese nicht verwertet werden. \cite{Payback_Teilnahme}
% * Kaufland
\section{Kaufland Card}
Die Kaufland Card ist eine Vorteilskarte für die Einzelhandelskette Kaufland. Die Kaufland Card existiert seit dem 28.10.2021 in Deutschland. Zuvor wurde diese schon in Osteuropa bereits 2019 eingeführt, dies betrifft z.B. die Länder Rumnänien, Tschechien, Slowakei oder auch Kroatien. 

\subsection{Ziel}
Kaufland hat mehrere Ziele und in Sachen der Digitalisierung bietet sie intelligente Lösungen wie digitale Obst- und Gemüse-Preisschilder, kontaktloses und mobiles bezahlen sowie die Kunden-App an. Über die Kunden App kann die Kaufland Card integriert werden. \cite{Kaufland_Ziele} \newline

\noindent \textit{``Die Nutzung der Kaufland Card zielt also darauf ab, dass die Teilnehmer relevantere Inhalte erhalten und Kaufland möglichst solche Informationen, die für den jeweiligen Teilnehmer nicht von Interesse sind, dem Teilnehmer gar nicht erst zur Verfügung stellt.'' \cite{Kaufland_Datenschutz}}

Das bedeutet Kaufland wertet das Kaufverhalten der Nutzer aus, um diesen gezielte Werbung präsentieren zu können.

\subsection{Rollen, Datenfreigabe und -verarbeitung}
Wie bei Payback, kristallisierten sich hier ebenfalls drei Rollen heraus. Zum einen der \textbf{Nutzer}, \textbf{Kaufland} selbst und die \textbf{Partnerunternehmen}. 
Um am Vorteilsprogramm, welches gekoppelt ist mit Extra-Rabatten, exklusiven Coupons, Treuepunkte und Gewinnspiele, teilnehmen zu können benötigt Kaufland personenbezogene Daten. Diese Daten sind:\newline
- die E-Mail Adresse oder die Mobilfunknummer\newline
- Geschlecht\newline
- Vorname \newline
- Nachname \newline
- Geburtsdatum \newline
- Adresse. \newline
Die einzige Möglichkeit die dem Nutzer bleibt ist zu entscheiden, ob er die Angabe für die Anzahl der im Haushalt lebenden Personen angeben möchte oder nicht. Kaufland gibt an, die geltenden Gesetzgebungen zum Schutze persönlicher Daten einzuhalten. Die Punkte die der Nutzer sammelt verfallen nach einem Jahr, insofern diese nicht eingelöst werden, dazu informiert Kaufland den Kunden per e-Mail, insofern man einer Kommunikation per Mail zugestimmt hat. \newline

\noindent Kaufland hat Partnerverträge mit Europcar, home24, Mister Spex und YogaEasy. \cite{Kaufland_FAQ} \newline

\noindent Kaufland schreibt vor, dass die Teilnehme nur privaten Nutzern vorbehalten ist, die über 18 Jahre sind und einen ständigen Wohnsitz in Deutschland oder den angrenzenden Ländern haben. Der Teilnehmer kann die Löschung der Kaufland Card und seines Kundenkontos selbst vornehmen. \cite{Kaufland_Datenschutz} \newline

\noindent Kaufland schreibt, das sie die Daten des Kunden der Schwarz Gruppe zur technischen Administration und statistischen anonymen Auswertung mitteilt. Weiterhin verfügt es über die Möglichkeit über einen Social-Login. Das bedeutet der Teilnehmer kann sich über seinen bereits bestehenden Social Media Account (z. B. Facebook, Google, Apple Account) auf den Webseiten oder der Kaufland App registrieren bzw. anmelden. Daten die vom Social Media Account an Kaufland übermittelt werden können sein: 
- Name
- Vorname
- Telefonnummer
- E-Mail Adresse
- Geburtsdatum. \newline

\noindent Auf der Kauflandseite steht geschrieben \textit{``Die Daten Ihres Kaufland-Kundenkontos liegen grundsätzlich nur im Zugriff der Fachbereiche innerhalb der Kaufland Gruppe, die mit der Pflege der Seite www.kaufland.de und der Kaufland-Kundenkonten beauftragt sind, bzw. die den konkret von Ihnen genutzten Kaufland Dienst anbieten. Eine Weitergabe an Dritte außerhalb der Kaufland-Gruppe erfolgt mit Ausnahme der dargestellten Daten an den Anbieter Ihres Social Media Accounts nicht.'' \cite{Kaufland_Rechtliches}} Weiterhin werden die Social Media Daten nur so lange genutzt bis der Teilnehmer von seinem Widerruf gebauch macht.

\subsection{Daten bei Kaufland monetarisieren}
Der monetare Wert für die Nutzerdaten lässt sich nicht in Zahlen bestimmen.
Der Nutzer hat die Möglichkeit Produkte, die in der Werbung ausgeschrieben sind günstiger zu bekommen. Diese sind in blauen Zahlen in der Werbung abgebildet. Das bedeutet Kaufland macht einen Unterschied von Karteninhabern und nicht Karteninhabern. Zum Beispiel ist in der Werbung eine Tüte Haribo ausgeschrieben für 89ct. Unter diesem Preis ist ein weiterer Preis abgebildet mit 69ct, welche nur für Kaufland Karteninhaber ist.
Weiterhin sammelt der Karteninhaber mit jedem Einkauf Punkte, insofern er seine Karte beim bezahlen einscannen lässt, damit Punkte gutgeschrieben werden können. 
Diese Punkte kann er gegen Prämien oder Coupons eintauschen.
Zusätzlich bekommt der Karteninhaber regelmäßig neue Coupons die er nutzen kann.
Zudem kann der Nutzer Vorteile bei Partnerunternehmen erhalten, welche Rabattcoupons darstellen. 
Die Punkte lassen sich nicht in Eurobeträge umwandeln und auf das Konto auszahlen.
Möchte der Nutzer nicht, dass von ihm Daten erhoben und ausgewertet werden, dann zeigt er keine Kaufland Karte beim bezahlen vor. Allerdings erhält er dafür keinen Gegenwert in Form von Punkten.

% * BitaAboutMe
\section{BitsAboutMe}
BitsAboutMe ist ein in der Schweiz ansässiges Unternehmen, welches eine Plattform bereitstellt, auf der man seine Daten verwalten und gleichzeitig verkaufen kann.

\subsection{Ziel}
Als Ziel setzt sich das Unternehmen, den Nutzern mehr Kontrolle über die eigenen Daten zu geben. Damit soll ein fairer Datenhandel zwischen Datenkonsumenten und Datenanbietern geschaffen werden, sowie eine aktive Teilnahme an der Monetariserung dieser. Weiterhin lädt das Unternehmen jedoch auch weitere Unternehmen ein, an diesem Netzwerk teilzunehmen und bietet unterschiedliche Analysen und Auswertungsmöglichkeiten, sowie eine Schnittstelle zu den Datenanbietern.

\subsection{Rollen bei BitsAboutMe}
\textbf{Nutzer:} Nutzer können auf der Plattform, ähnlich wie im Abschnitt \ref{invisibly} bei Invisibly, ihre Daten von diversen Konten, wie Instagram, Facebook, Spotify, Amazon aber auch E-Mail- und Bankkonten hinzufügen. Neben dem Verkauf von persönlichen Daten soll ein Überblick über die Daten des Nutzers an einem zentralen Ort verschafft werden. Der Nutzer kann durch den Import verschiedener Datenquellen herausfinden, wie das eigene Nutzungsverhalten in sozialen Netzwerken ist, wie groß der eigene CO\textsubscript{2}-Fußabdruck ist oder ob man einen gesunden Lifestyle führt. Um damit nur einige Möglichkeiten zu nennen, hat der Nutzer also damit die Möglichkeit die Daten der jeweiligen Datenquellen einzusehen und aggregiert darzustellen und sich auswerten zu lassen. \newline

\noindent Der Nutzer fungiert demzufolge wiederum als Datenanbieter beziehungsweise Datenquelle, wobei jedoch die Privatsphäre im Vordergrund steht. BitsAboutMe garantiert durch eine Verschlüsselung der Datenbank auf Nutzerebene und einem Rechenzentrum in der EU oder in der Schweiz für eine hohe Sicherheit und ausschließliche Einsicht der Daten durch den Nutzer selbst. Dieser kann mithilfe der \gls{PWA} stets selbst entscheiden, welche Daten konkret freigegeben werden und welche nicht.\newline

\begin{figure}[!htbp]
	\centering
	\includegraphics[width=\textwidth]{bitsaboutme_datenquellen}
	\caption{Möglichkeiten des Hinzufügens von Datenquellen auf der BitsAboutMe-Plattform mit ungefährer Zeitabschätzung}
	\label{fig:bitsaboutmeDatenquellen}
\end{figure}
\FloatBarrier

\noindent \textbf{Käufer:} Käufer auf der BitsAboutMe-Plattform sind Unternehmen, welchen BitsAboutMe eine SaaS-Lösung anbietet, mit der sie als Datenkonsumenten die bereitsgestellten Daten von Nutzern erwerben können und Überblick über stattgefundene Trankaktionen erhalten.

\subsection{Daten auf der BitsAboutMe-Plattform monetarisieren}
Damit der Nutzer die persönlichen Daten nun monetarisieren kann, bietet er eine Auswahl der Daten nun auf dem Daten-Marktplatz an. BitsAboutMe verfolgt dabei eine getrennte Architektur vom persönlichen Datenspeicher (im Folgenden \textit{PDS} genannt) und dem persönlichen Daten-Marktplatz (im Folgenden \textit{PDM} genannt). Das bedeutet, dass die Daten bei dem Verknüpfen der Konten mit der BitsAboutMe-Plattform vorerst verschlüsselt im PDS abgelegt werden, worauf der Nutzer alleinig zugreifen kann. Entschließt dieser sich nun zu einem Verkauf dieser, so werden diese anonymisiert und an den PDM transferiert, sodass durch den Nutzer ausgewählte Datenkonsumenten beziehungsweise Unternehmen Zugriff auf das Datenprofil haben. Durch das Teilen des Datenprofils verdient der Nutzer Geld. Zusätzlich erhält BitsAboutMe ebenfalls eine Gebühr für die Transaktion.\newline

\noindent Möchte der Nutzer jedoch zukünftig nicht mehr sein anonymisiertes Datenprofil zur Verfügung stellen, so ist eine Rücknahme oder eine Löschung des Datenprofils jederzeit möglich. \newline

\begin{figure}[h!]
	\centering
	\includegraphics[width=\textwidth]{bitsaboutme_marktplatz.jpg}
	\caption{Persönlicher Datenmarktplatz auf der BitsAboutMe-Plattform mit Angeboten, angenommen Angeboten und Informationen zu abgerufenen Daten}
	\label{fig:bitsaboutmeMarktplatz}
\end{figure}
\FloatBarrier

\noindent Neben dem Teilen des Nutzerprofils bietet die Plattform noch eine weitere Möglichkeit Daten anonym zu monetarisieren. Undzwar kann der Nutzer von ihm erhaltene Kassenzettel einscannen und 1\% Cashback auf den jeweiligen Einkauf erhalten. Dies ist jedoch auf 10€ im Monat begrenzt. Diese Daten fließen ebenfalls in das persönliche Benutzerprofil ein, da BitsAboutMe anhand dessen analysiert, wie nachhaltig und gesund der Nutzer lebt und gleichzeitig einen Überblick über die Finanzen schafft. Damit die Daten valide sind und damit brauchbar für die Datenkonsumenten, werden die Kassenbons mit den Zahlungen des hinterlegten Bankkontos abgeglichen. Nur auf Zahlungen, die auf dem Bankkonto durchgeführt wurden, gibt es Cashback. Unternehmen können diese Daten folglich erwerben und für die Marktforschung und andere, im Abschnitt \ref{datenoekonomie} Datenökonomie, beschriebene Zwecke nutzen. \newline

\noindent Zusammengefasst sind folgende Schritte für die Monetariserung der Daten notwendig:
\begin{enumerate}
	\item Registieren
	\item Datenquellen verbinden
	\item (digitale Identität prüfen)
	\item PDM aktivieren
	\item Daten freigeben und Geschäft abschließen
\end{enumerate}

% * Datum
\section{Datum}
Datum ist ein Netzwerk, indem Nutzer ihre Daten dezentralisiert in einer Blockchain speichern können. Darüberhinaus ermöglicht es dem Nutzer diese Daten zu kaufen oder verkaufen und die Nutzung dieser entsprechend einzuschränken. Es ist also nicht nur ein Speicherort für Daten, sondern dient auch als Online-Datenmarktplatz und \gls{brokerG} zugleich.

\subsection{Ziel}

\subsection{Rollen im Datum-Netzwerk}
\textbf{Nutzer:} Nutzer sind diejenigen Personen im Netzwerk, die ihre persönlichen oder geschäftlichen Daten hinterlegen und zum Verkauf anbieten können. Sie nehmen dabei also die Rolle des Datenanbieters ein. Datum ermöglicht dem Nutzer zudem eine Einflussnahme auf die Privatsphäreeinstellungen, sodass folgende fünf Einstellungen getroffen werden können: 
\begin{enumerate}
	\item Das Teilen der Daten ist nicht erlaubt
	\item Das Teilen der Daten ist nur mit ganz bestimmten, identifizierten und dem Nutzer bekannten Datenkonsumenten gestattet
	\item Das Teilen der Daten ist wiederum nur mit ganz bestimmten, identifizierten und dem Nutzer bekannten Datenkonsumenten gestattet aber für eine Mindestgebühr
	\item Die Daten sind für jeden verfügbar
	\item Die Daten sind für jeden verfügbar aber für eine Mindestgebühr
\end{enumerate}

\noindent \textbf{Käufer:} Käufer können sich im Datum-Netzwerk den Zugriff auf die Daten der Nutzer erkaufen. Sie fungieren in diesem Netzwerk demzufolge als Datenkonsumenten, wobei sie jedoch nur einen eingeschränkten Zugriff auf die Daten haben, und zwar entsprechend der Nutzungsbedingungen des Nutzers. Auch sie können unterschiedliche Informationen offenlegen:
\begin{enumerate}
	\item Identität des Käufers
	\item Die allgemeine Datenschutzerklärung
	\item Die Zweckmäßigkeit
	\item Die Dauer der Aufbewahrung der Daten
	\item Das \textit{Datum-Network-Trust-Rating}
\end{enumerate}

\noindent \textbf{DAT-Token-Holder:} DAT-Token-Holder steuern das Netzwerk und ermöglichen Trankaktionen in dem Netzwerk. \newline

\noindent \textit{Hinweis: Auf weitere Rollen, wie beispielsweise Storage Nodes wird im Folgenden nicht eingegangen, da sich diese vorwiegend mit der technischen Komponente einer Blockchain befassen und im Sinne der Datenökonomie weniger relevant sind.}

\subsection{Daten im Datum-Netzwerk monetarisieren}

\subsection{Probleme}
% * Invisibly
\section{Invisibly} \label{invisibly}
Invisibly ist eine Plattform, bei der Benutzer ihre persönlichen Daten zur lizenzierten Freigabe bereitstellen können. Sie befindet sich derzeit noch in der Beta-Phase und ist bisher ausschließlich in Amerika verfügbar.

\subsection{Hintergrund und Ziel}
Die Plattform hatte ursprünglich das Ziel, bezahlungspflichtige Artikel leichter zugänglich zu machen. Benutzer der Webseite sollten Werbung ansehen und dafür Token erhalten, welche sie anschließend für den Zugang zu einzelnen Publikationen von Nachrichtenseiten und Wissenschaftsmagazinen ausgeben konnten -- ohne bei den einzelnen Diensten beispielsweise ein Abonnement abzuschließen. Da diese Idee nur wenige Personen überzeugte, wurde das Konzept überarbeitet. \cite{techRadarInvisibly_2021} \newline

\noindent Heute verfolgt Invisibly das Ziel, seine Benutzer mit einem fairen Anteil am Wert zu beteiligen, der bei der Verarbeitung persönlicher Daten entsteht. Die Plattform versucht dies aktuell auf zwei verschiedene Wege. Im Gegensatz zu anderen Plattformen, die Inhalte strategisch platzieren, versucht Invisibly als erstes seinen Benutzern auf Basis der freiwillig geteilten Daten hochwertige Inhalte zu präsentieren, die sie tatsächlich sehen wollen: \begin{quote}
    \textit{``We are creating an AI-powered platform with a feed that`s a true extension of you, rather than a feed strategically curated and targeted at you by the Big Tech and brands.'' \cite{invisiblyWhyPay_2021}}
\end{quote} Im zweiten Schritt zielt Invisibly darauf ab, seine Benutzer direkt am Verkauf der persönlichen Daten zu beteiligen, indem sie ihnen regelmäßig Geld auszahlt. \cite{invisiblyWhyPay_2021} Der Gründer von Invisibly, Jim McKelvey, beschreibt seine Plattform in einem Interview wiefolgt: \begin{quote}
    \textit{``We basically act as your agent. We try to sell you to advertisers and we give you the money. We ask how much you are willing to sell, package it up and sell it to the highest bidder.'' \cite{techRadarInvisibly_2021}}
\end{quote} Die Plattform verkauft die Rohdaten ihrer Nutzer jedoch nicht direkt -- stattdessen lizenziert Invisibly die Daten ihrer Nutzer. Auf diese Weise bleibt die individuelle Person Eigentümer der Daten und kann selbst bestimmen, welche Daten sie freigeben möchte. \cite{invisiblyGetPaid_2021} Anschließend vermittelt Invisibly die Lizenzen an Werbetreibende für Dienste und Produkte, die am besten zum jeweiligen Benutzer passen. Sie erhalten mit einer Lizenz solange Zugriff auf die persönlichen Daten, bis ein Benutzer seine Lizenz widerruft und somit die Freigabe beendet. Der Vorteil für Unternehmen besteht darin, dass Invisibly sie durch den Kauf von Lizenzen mit den Individuen zusammenbringt, die potentiell am besten zu deren Produkten passen. \cite{techRadarInvisibly_2021} Dabei ist es wichtig zu erwähnen, dass Benutzer von Invisibly ihre Daten freiwillig zur Lizenzierung freigeben und deshalb wahrscheinlich von einer Vermittlung an Dritte nicht abgeneigt sind. \newline

\noindent Im Folgenden wird erklärt, wie Individuen ihre persönlichen Daten bei Invisibly zu Geld machen können.

\subsection{Daten monetarisieren}
Benutzer sammeln bei Invisibly Punkte, wenn sie persönliche Daten mit der Plattform teilen. Gesammelte Punkte können später in US-Dollar umgerechnet und ausgezahlt werden. Derzeit gibt es drei verschiedene Möglichkeiten, Punkte zu sammeln: \newline

\noindent \textbf{Datenquellen verlinken:} Die meisten Punkte sammeln Benutzer, indem sie verschiedene Datenquellen in ihrem Invisibly-Profil hinterlegen. Für jede Datenquelle wird monatlich ein fester Betrag an Punkten gutgeschrieben, ähnlich wie bei einem passiven Einkommen. \cite{pymntsInvisibly_2021} Das Verknüpfen eines Bankkontos wird beispielsweise mit \textit{75 Punkten} pro Monat vergütet, wobei Werbepartner so Zugriff auf sämtliche Transaktionsdaten erhalten. Andererseits lassen sich verschiedene soziale Netzwerke bei Invisibly hinterlegen. Für jeden Account erhält ein Benutzer monatlich \textit{25 Punkte} und es werden aktuell die Plattformen Instagram, Twitter, TikTok, LinkedIn und Pinterest unterstützt. Zum Schluss bietet Invisibly eine Browser-Erweiterung an, welche den Verlauf der besuchten Webseiten aufzeichnet. Mit ihr sammeln Benutzer \textit{200 Punkte} pro Monat. \cite{instagramInvisibly_2021, lifewireInvisibly_2021} Abbildung \ref{fig:invisiblyProfile} zeigt im zweiten Screenshot den Tab \textit{Data Vault} der Invisibly-App, in welchem die hinterlegten Datenquellen verwaltet werden können. \newline

\begin{figure}[!ht]
	\centering
	\includegraphics[width=\textwidth]{invisibly_profile}
	\caption{Ansichten in der mobilen Invisibly-App: My Opportunities, Data Vault und Wallet \cite{behanceInvisibly_2021}}
	\label{fig:invisiblyProfile}
\end{figure}

\noindent \textbf{Fragen zur Person:} Benutzer können ihr Invisibly-Profil vervollständigen, indem sie verschiedene Fragen zu ihrer Person beantworten. So fragt die Webseite beispielsweise nach dem Geschlecht, dem höchsten Bildungsabschluss oder ob man Kinder hat. Jede Antwort wird dabei mit \textit{einem Punkt} vergütet. \cite{instagramInvisibly_2021} \newline

\noindent \textbf{Persönliches Feed:} Das persönliche Feed ist Invisibly's Vision von einer besseren Alternative zu Facebook und ähnlichen Beispielen. Das Feed enthält relevante Beiträge, die auf Basis der freiwillig geteilten, persönlichen Daten von verschiedenen Datenquellen ausgewählt und angezeigt werden. Diese Beiträge bestehen aus Bildern mit kurzen Texten und verweisen auf Diensteleistungen und Produkte von Werbetreibenden, wie beispielsweise Gegenstände zum Kaufen, Reise- und Ausflugsziele, Veranstaltungen, Gutscheine oder sonstige Angebote in Online-Shops und vieles mehr. Jeder Beitrag kann mit einem \textit{Like} oder \textit{Dislike} markiert werden -- auf diese Weise sammelt Invisibly Daten über persönliche Interessen und kann Vorschläge noch besser auf jede Person zuschneiden. Benutzer erhalten im Gegenzug für jeden Like oder Dislike \textit{einen Punkt}, wobei die maximale Anzahl an Punkten hier auf 20 Punkte pro Tag begrenzt ist. \cite{invisiblyWhyPay_2021} \newline

\noindent Das in Abbildung \ref{fig:invisiblyProfile} gezeigte erste Screenshot der Invisibly-App zeigt die Ansicht \textit{My Opportunities}, in welcher dem Benutzer noch nicht erledigte Möglichkeiten zum Sammeln von Punkten vorgeschlagen werden. An erster Stelle werden die Datenquellen aufgelistet, für welche noch kein Account hinterlegt ist, gefolgt von Fragen zur Person und dem persönlichen Feed. \newline

\noindent Die gesammelten Punkte können abschließend direkt in Geld umgewandelt werden: 100 Punkte entsprechen \$1 und eine Auszahlung ist ab 500 Punkten möglich. \cite{invisiblyWhyPay_2021} Eine Auszahlung ist über die \textit{Wallet} möglich, welche sowohl in der mobilen Version als auch über die in Abbildung \ref{fig:invisiblyWallet} dargestellte Webseite aufrufbar ist. Hier können Bentuzer neben ihrem aktuellen Punktestand auch einen Verlauf der gutgeschriebenen Punkte einsehen. Eine Auszahlung ist derzeit ausschließlich über den amerikanischen Zahlungsdienstleister PayPal möglich. \newline

\begin{figure}[!ht]
	\centering
	\includegraphics[width=0.9\textwidth]{invisibly_wallet}
	\caption{Wallet mit Kontostand und Verlauf der gesammelten Punkte \cite{behanceInvisibly_2021}}
	\label{fig:invisiblyWallet}
\end{figure}

\noindent Die Monetarisierung von persönlichen Daten bringt Invisibly's Nutzern derzeit ca. \$60 bis \$100 im Jahr, abhängig von der Anzahl verlinkter Datenquellen. Das liegt daran, dass aktuell ein flaches Bezahlmodell verwendet wird, bei dem jeder Nutzer gleich viel Geld bzw. Punkte für seine persönlichen Daten bekommt -- unabhängig davon, wieviel die Daten konkret für Werbetreibende wert sind. In Zukunft soll ein kompetitives Modell zum Einsatz kommen, bei dem Personen eine höhere Dividende erhalten, wenn die Daten für Werbetreibende mehr wert sind. \cite{pymntsInvisibly_2021} Der Gründer von Invisibly geht davon aus, dass ein schnell wachsender Marktplatz mit mehr Werbetreibenden und Individuen in wenigen Jahren bereits \$1000 pro Jahr für Nutzer des Dienstes erwirtschaften kann. \cite{techRadarInvisibly_2021} 


\chapter{Diskussion und Ergebnisse}
\section{Ergebnisse}

\begin{table}[!ht]
\begin{tabular}[h]{ |l|c|c|c|c|c| }
    \hline
    \thead{} & \thead{Cashback} & \thead{Payback} & \thead{BitsAboutMe} & \thead{Datum} & \thead{Invisibly} \\
    \hline
    \makecell{\textbf{Transparenz\textsuperscript{1}}} & & & & & \\
    \makecell{Konsument} & & & \cmark & \cmark & \\
    \makecell{Zeitraum} & & & \cmark & \cmark & \\
    \makecell{Verwendungszweck} & & & \cmark & \cmark & \\
    \makecell{Umfang} & & & \cmark & \cmark & \\
    \hline
    \makecell{\textbf{aktive Teilnahme\textsuperscript{2}}} & & & & & \\
    \makecell{bewusste Freigabe} & & & \cmark & \cmark & \\
    \makecell{Rückruf möglich} & & & \cmark & \cmark & \\
    \hline
    \makecell{\textbf{Gegenwert\textsuperscript{3}}} & & & & & \\
    \makecell{Geld} & & & \cmark & \cmark & \\
    \makecell{Preisnachlass} & & & \xmark & \xmark & \\
    \makecell{Prämien} & & & \xmark & \xmark & \\
    \hline
    \makecell{\textbf{Datenerhebung\textsuperscript{4}}} & & & & & \\
    \makecell{First-Party} & & & \xmark & \xmark & \\
    \makecell{Second-Party} & & & \xmark & \xmark & \\
    \makecell{Third-Party} & & & \cmark & \cmark & \\
    \hline
    \makecell{\textbf{Geschäftsmodell\textsuperscript{5}}} & & & & & \\
    \makecell{Datennutzer} & & & \xmark & \xmark & \\
    \makecell{Datenlieferanten} & & & \xmark & \xmark & \\
    \makecell{Datenvermittler} & & & \cmark & \cmark & \\
\end{tabular}
\caption{\label{tab:Auswertung der Fallstudie} Übersicht zur Auswertung der Fallstudie.}
\end{table}

\noindent \textsuperscript{1}Der Nutzer hat Einsicht über Art und Umfang der verarbeiteten Daten - unterteilt in nachfolgende Kategorien \newline

\noindent \textsuperscript{2}Der Nutzer nimmt aktiv an der Monetariserung seiner Daten teil - unterteilt in nachfolgende Kategorien \newline

\noindent \textsuperscript{3}Der Gegenwert, den der Nutzer für den Verkauf seiner Daten erhält - unterteilt in nachfolgende Kategorien\newline

\noindent \textsuperscript{4}Die Art der Datenerhebung aus Sicht des Datenkonsumenten gemäß Kapitel \ref{datenoekonomie} Datenökonomie - unterteilt in nachfolgende Kategorien\newline

\noindent \textsuperscript{5}Das Geschäftsmodell des Unternehmens gemäß Kapitel \ref{datenoekonomie} Datenökonomie - unterteilt in nachfolgende Kategorien\newline

\chapter{Zusammenfassung}

% ==============================================
% Literaturverzeichnis
% ==============================================
\nocite{*}
\printbibliography
\addcontentsline{toc}{chapter}{Literaturverzeichnis}

% ==============================================
% Anhang
% ==============================================
\appendix
\chapter*{Anlagen} \label{Anlagen}
\addcontentsline{toc}{chapter}{Anlagen}
\setcounter{chapter}{1} 

\begin{figure}[!ht]
\centering
\includegraphics[width=\textwidth]{Bilddatei}
\caption{Bildbeschreibung}
\label{fig:Bildlabel}
\end{figure}

% Endblatt (weiß)
\clearpage
\thispagestyle{empty}
\begin{center}
\vfill
\phantom{© \makeatletter\@author\makeatother}
\end{center}

\end{document}
