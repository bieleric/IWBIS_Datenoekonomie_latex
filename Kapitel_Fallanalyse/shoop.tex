\section{Shoop}
Die Plattform Shoop, ist ein gebührenfreies Cashback System mit Sitz in Deutschland für Privatanwender. Mit welchem Nutzer keine Treuepunkte sammeln, sondern direktes Geld auf ihr Konto zurück erhalten. 

\subsection{Ziel}
Das Ziel von Shoop ist es durch die die Einkäufe von Nutzern Provisionen zu generieren und davon einen Teil an die Nutzer zurück zu geben. \cite{shoop}
Dazu baut Shoop auf das System von Affiliate Marketing auf. Dies ist ein Marketing Kanal im Online Marketing, um Kunden zu werben. \cite{affiliate_marketing}

\subsection{Rollen, Datenfreigabe und -verarbeitung}
Shoop benennt die Rollen des \textbf{Onlinehändlers}, \textbf{Publisher} und \textbf{den Nutzer}.
Die Onlinehändler sind die Verkäufer die Kunden gewinnen und Produlte verkaufen wollen. Die Publisher sind z. B. Blogger die Werbung für die Onlinehändler machen. Die Nutzer sind die Kunden die über, zum Beispiel Affiliate Links von Bloggern einen Einkauf beim Onlinehändler tätigen. Somit ist auch Shoop ein Publisher und bietet über seine Seite Verlinkungen zu Onlinehändlern an. \cite{affiliate_marketing}

Shoop gibt an, dass sie keinen Datenhandel betreibt und somit geben diese auch keine Daten an Dritte weiter. \cite{shoop}
Allerdings benötigen sie für die Registrierung eine e-Mail Adresse und einen Nutzernamen. Der Nutzer kann sich aber auch mit seinem Facebook-Account anmelden. Dazu werden Profildaten  und äffentliche Daten von Facebook an Shoop übermittelt und umgekehrt,dies dient zum Zweck der Registrierung.
Shoop erhebt nur Daten vom Nutzer die im Zusammenhang mit der Teilnahme am System erforderlich sind. Shoop benötigt persönliche Daten und die Bankverbindung, um das Geld auszahlen zu können. Der Nutzer kann jederzeit Einsicht in seine Aktivitäten nehmen und den derzeitigen Account-Stand erfahren. Für die erfolgreiche Abwicklung existiert ein Tracking-Prozess, dazu gehört der Tracking-Link (Affiliate-Link) der aufgerufen wird, wenn der Link angeklickt wurde.
Der Nutzer hat zugleich jederzeit das Recht seine Teilnahme durch Löschung seines Accounts zu kündigen.\cite{shoop_agb}

\subsection{Daten bei Shoop monetarisieren}
Bei Shoop lassen sich keine persönlichen Daten monetarisieren. Allerdings erhält der Nutzer einen Prozentsatz, abhängig vom Einkaufswert und von der Cashback-Rate des Onlinehändlers,zurück. 