\section{Kaufland Card}
Die Kaufland Card ist eine Vorteilskarte für die Einzelhandelskette Kaufland. Die Kaufland Card existiert seit dem 28.10.2021 in Deutschland. Zuvor wurde diese schon in Osteuropa bereits 2019 eingeführt, dies betrifft z.B. die Länder Rumnänien, Tschechien, Slowakei oder auch Kroatien. 

\subsection{Ziel}
Kaufland hat mehrere Ziele und in Sachen der Digitalisierung bietet sie intelligente Lösungen wie digitale Obst- und Gemüse-Preisschilder, kontaktloses und mobiles bezahlen sowie die Kunden-App an. Über die Kunden App kann die Kaufland Card integriert werden. \cite{Kaufland_Ziele} \newline

\noindent \textit{``Die Nutzung der Kaufland Card zielt also darauf ab, dass die Teilnehmer relevantere Inhalte erhalten und Kaufland möglichst solche Informationen, die für den jeweiligen Teilnehmer nicht von Interesse sind, dem Teilnehmer gar nicht erst zur Verfügung stellt.'' \cite{Kaufland_Datenschutz}}

\subsection{Datenfreigabe und -verarbeitung}
Um am Vorteilsprogramm, welches gekoppelt ist mit Extra-Rabatten, exklusiven Coupons, Treuepunkte und Gewinnspiele, teilnehmen zu können benötigt Kaufland personenbezogene Daten. Diese Daten sind  die E-Mail Adresse oder die Mobilfunknummer, Geschlecht, Vorname, Nachname, Geburtsdatum und die Adresse. Die einzige Möglichkeit die dem Kunden bleibt ist zu entscheiden, ob er die Angabe für die Anzahl der im Haushalt lebenden Personen angeben möchte oder nicht. Kaufland gibt an die geltenden Gesetzgebungen zum Schutze persönlicher Daten einzuhalten. Die Punkte die der Kunde sammelt verfallen nach einem Jahr insofern diese nicht eingelöst werden, dazu informiert Kaufland den Kunden per e-Mail, insofern man einer Kommunikation per Mail zugestimmt hat. \newline

\noindent Kaufland hat Partnerverträge mit Europcar, home24, Mister Spex und YogaEasy. \cite{Kaufland_FAQ} \newline

\noindent Kaufland schreibt vor das die Teilnehmer nur privaten Nutzern vorbehalten ist, die über 18 Jahre sind und einen ständigen Wohnsitz in Deutschland oder den angrenzenden Ländern haben. Der Teilnehmer kann die Löschung der Kaufland Card und seines Kundenkontos selbst vornehmen. \cite{Kaufland_Datenschutz} \newline

\noindent Kaufland schreibt das sie die Daten des Kunden der Schwarz Gruppe zur technischen Administration und statistischen anonymen Auswertung mitteilt. Weiterhin bietet es die Möglichkeit über einen Social-Login. Das bedeutet der Teilnehmer kann sich über seinen bereits bestehenden Social Media Account (z. B. Facebook, Google, Apple Account) auf den Webseiten oder der Kaufland App registrieren bzw anmelden. Daten die vom Social MEdia Account an Kaufland übermittelt werden können sein: Name, Vorname, Telefonnummer, E-Mail Adresse, Geburtsdatum. \newline

\noindent Auf der Kauflandseite steht geschrieben \textit{``Die Daten Ihres Kaufland-Kundenkontos liegen grundsätzlich nur im Zugriff der Fachbereiche innerhalb der Kaufland Gruppe, die mit der Pflege der Seite www.kaufland.de und der Kaufland-Kundenkonten beauftragt sind, bzw. die den konkret von Ihnen genutzten Kaufland Dienst anbieten. Eine Weitergabe an Dritte außerhalb der Kaufland-Gruppe erfolgt mit Ausnahme der dargestellten Daten an den Anbieter Ihres Social Media Accounts nicht.'' \cite{Kaufland_Rechtliches}} Weiterhin werden die Social Media Daten nur so lange genutzt bis der Teilnehmer von seinem Widerruf gebauch macht.