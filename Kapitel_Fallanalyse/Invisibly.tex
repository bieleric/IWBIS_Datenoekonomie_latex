\section{Invisibly}

Invisibly ist eine Plattform, bei der Benutzer ihre persönlichen Daten zur lizenzierten Freigabe bereitstellen können und dauerhaft einen monetären Gegenwert erhalten.

\subsection{Ziel}
\cite{techRadarInvisibly_2021}

\subsection{Daten monetarisieren}
Benutzer sammeln bei Invisibly Punkte, welche sie in Geld umrechnen und auszahlen lassen können. Derzeit gibt es drei verschiedene Möglichkeiten, Punkte zu sammeln: \newline

\noindent \textbf{Datenquellen verlinken:} Die meisten Punkte sammeln Benutzer, indem sie verschiedene Datenquellen in ihrem Invisibly-Profil hinterlegen. Für jede Datenquelle wird monatlich ein fester Betrag an Punkten gutgeschrieben, ähnlich wie bei einem passiven Einkommen. \cite{pymntsInvisibly_2021} Das Verknüpfen eines Bankkontos wird beispielsweise mit \textit{75 Punkten} pro Monat vergütet, wobei Werbepartner so Zugriff auf sämtliche Transaktionsdaten erhalten. Andererseits lassen sich verschiedene soziale Netzwerke bei Invisibly hinterlegen. Für jeden Account erhält ein Benutzer monatlich \textit{25 Punkte} und es werden aktuell die Plattformen Instagram, Twitter, TikTok, LinkedIn und Pinterest unterstützt. Zum Schluss bietet Invisibly eine Browser-Erweiterung an, welche den Verlauf der besuchten Webseiten aufzeichnet. Mit ihr sammeln Benutzer \textit{200 Punkte} pro Monat. \cite{instagramInvisibly_2021, lifewireInvisibly_2021} \newline

\noindent \textbf{Fragen zur Person:} Benutzer können ihr Invisibly-Profil vervöllständigen, indem sie verschiedene Fragen zu ihrer Person beantworten. So fragt die Webseite beispielsweise nach dem Geschlecht, dem höchsten Bildungsabschluss oder ob man Kinder hat. Jede Antwort wird dabei mit \textit{einem Punkt} vergütet. \cite{instagramInvisibly_2021} \newline

\noindent \textbf{Persönliches Feed:} Außerdem wird jedem Benutzer ein Feed mit Beiträgen angezeigt, welche für die jeweilige Person relevant sein könnten. Diese Beiträge bestehen aus Bildern mit kurzen Texten und verweisen auf Diensteleistungen und Produkte von Dritten, wie beispielsweise Gegenstände zum Kaufen, Reise- und Ausflugsziele, Veranstaltungen, Gutscheine oder sonstige Angebote in Online-Shops und vieles mehr. Jeder Beitrag kann mit einem \textit{Like} oder \textit{Dislike} markiert werden -- auf diese Weise sammelt Invisibly Daten über persönliche Interessen und kann Vorschläge noch besser auf jede Person zuschneiden. Benutzer erhalten im Gegenzug für jeden Like oder Dislike \textit{einen Punkt}, wobei die maximale Anzahl an Punkten hier auf 20 Punkte pro Tag begrenzt ist. \cite{invisiblyWhyPay_2021} \newline

\noindent Die gesammelten Punkte können abschließend direkt in Geld umgewandelt werden: 100 Punkte entsprechen 1 US-Dollar und eine Auszahlung ist ab 500 Punkten möglich. \cite{invisiblyWhyPay_2021}