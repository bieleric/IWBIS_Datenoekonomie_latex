\section{Kategorisierung der Dienste}
Die Ergebnisse der Fallstudie deuten darauf hin, dass zwischen den Plattformen Gemeinsamkeiten bestehen, mit denen sich Kategorien bilden lassen. Einerseits können die Dienste nach der \textit{Art der Daten} unterteilt werden, welche Individuen monetarisieren. Andererseits lassen sich die Plattformen nach dem \textit{Gegenwert} kategoriesieren, den Personen für die Bereitstellung persönlicher Daten erhalten. Abschließend unterscheiden sich die Dienste auch in ihrer \text{Funktionsweise} sowie \textit{Kontrolle} und \textit{Transparenz} für Benutzer. \newline

\noindent \textbf{Unterscheidung nach Art der Daten:} Bei dieser Kategorisierung wird zwischen den Daten unterschieden, die Individuen aktiv mit der Plattform teilen. Die Ergebnisse zeigen, dass es hier zwei Kategorien gibt. Die erste Gruppe lässt sich unter dem Begriff \textit{Konsumdaten} zusammenfassen und beinhaltet die beiden Anbieter Payback und Kaufland Card. Der Fokus dieser Dienste liegt darauf, dass Individuen Daten zu ihrem Konsumverhalten zur Datenanalyse freigeben und dafür entlohnt werden. Die Dienste Invisibly und BitsAboutMe lassen sich hier auch teilweise zuordnen, da bei ihnen Bankkonten hinterlegt bzw. Kassenzettel hochgeladen werden können. In der zweiten Gruppe befinden sich Dienste, auf denen Personen sämtliche \textit{Daten zum persönlichen Profil} monetarisieren können. Als Datenquellen dienen hier insbesondere Social Media Accounts und eigene Angaben zur Person. In diese Gruppe fallen die Anbieter BitsAboutMe, Datum und Invisibly.
\begin{itemize}
    \item Konsumdaten: Payback, Kaufland Card, BitsAboutMe, Invisibly
    \item Persönliches Profil: BitsAboutMe, Datum, Invisibly
\end{itemize}

\noindent \textbf{Unterscheidung nach Gegenwert:} Beim Gegenwert bilden sich drei verschiedene Gruppen zur Kategorisierung. Die erste Gruppe umfasst alle Dienste, bei denen Individuen \textit{bares Geld} für ihre persönlichen Daten erhalten. Darunter fallen die Anbieter BitsAboutMe, Datum und Invisibly sowie Payback durch seine Auszahlungsfunktion. Die zweite Gruppe beinhaltet alle Dienste, auf denen Personen einen \textit{monetäre Gegenwert} erhalten. Damit sind Gutscheine, Rabatte und Prämien gemeint, da diese sich in sofern von Bargeld unterscheiden, dass sie beim Einlösen an Bedingungen wie beispielsweise bestimmte Partner gebunden sind. Zu dieser Gruppe zählen Payback und Kaufland Card. Die letzte Gruppe beinhaltet alle Anbieter, bei denen Personen \textit{Dienstleistungen} für ihre persönlichen Daten erhalten. Darunter fallen BitsAboutMe mit den Analyseservices, Invisibly mit dem persönlichen Feed für Benutzer sowie Payback und Kaufland Card mit personalisierter Werbung.
\begin{itemize}
    \item Bares Geld: Payback, BitsAboutMe, Datum, Invisibly
    \item Geldwerte: Payback, Kaufland Card
    \item Dienstleistungen: Payback, Kaufland Card, BitsAboutMe, Invisibly
\end{itemize}

\noindent \textbf{Unterscheidung nach Funktionsweise:} Die Recherche zu den Anbietern hat gezeigt, dass Dienste zur Monetarisierung von Konsumdaten bereits unter dem Begriff \textit{Bonusprogramme} zusammengefasst werden. Diese unterteilen sich in die beiden Gruppen Payback Programme, bei denen Personen Gutscheine oder Rabatte erhalten, und Cashback Programmen, bei denen Personen bares Geld ausgezahlt bekommen. Payback Programme meint in diesem Kontext sämtliche Anbieter und nicht das Unternehmen Payback selbst. Zu den Bonusprogrammen zählen die betrachteten Plattformen Payback und Kaufland Card. Darüber lassen sich Dienste zusammenfassen, die als \textit{Marktplatz} agieren. Auf ihnen werden Individuen als Datenanbieter mit Werbetreibenden und anderen Dritten als Datenkonsumenten in Verbindung gebracht, wobei Daten entweder verkauft oder zur Nutzung lizenziert werden. Zu den Marktplätzen zählen BitsAboutMe, Datum und Invisibly.
\begin{itemize}
    \item Bonusprogramm: Payback, Kaufland Card
    \item Marktplatz: BitsAboutMe, Datum, Invisibly
\end{itemize}

\noindent \textbf{Unterscheidung nach Kontrolle:} Bezüglich Kontrolle bestehen deutliche Unterschiede zwischen den Diensten, weshalb hier eine Abstufung zwischen \textit{hoher}, \textit{mittlerer} und \textit{geringer} Kontrolle und Transparenz stattfindet. Hohe Kontrolle über persönliche Daten bieten die Anbieter BitsAboutMe und Datum durch Funktionen wie beispielsweise nachträgliches Löschen einzelner Datensätze oder eigene Selektion der Datenkäufer und welche Daten verkauft werden. Mittlere Kontrolle haben Benutzer bei Payback und Invisibly. Bei Payback besteht die Möglichkeit, der Marktforschung und Werbenutzung insgesamt zu widersprechen und auf Invisibly können Anwender jederzeit gesamte Datenquellen entfernen. Eine niedrige Kontrolle über persönliche Daten ist bei Kaufland Card gegeben, da hier nur die Löschung des Benutzerkontos möglich ist.
\begin{itemize}
    \item Hoch: BitsAboutMe, Datum
    \item Mittel: Payback, Invisibly
    \item Gering: Kaufland Card
\end{itemize}

\noindent \textbf{Unterscheidung nach Transparenz:} Eine ähnliche Unterscheidung findet hinsichtlich Transparenz für Individuen statt. Hier wird jedoch nur zwischen \textit{hoher} und \textit{niedriger} Transparenz bewertet. Eine hohe Transparenz erhalten Benutzer bei BitsAboutMe und Datum, da hier stets aufgeklärt wird, welcher Datenkonsument welche Datensätze erhält und wie er diese Daten vertraglich verwendet. Die Dienste Invisibly, Payback und Kaufland Card hingegen haben alle eine geringe Transparenz über die Verarbeitung der Daten. Einerseits wird der Verkauf an Dritte und der Adresshandel mit Partnern bei Payback ausgeschlossen. Allerdings ist nicht bekannt, wie die internen Profile über ihre Nutzer zustande kommen und welche persönlichen Daten zu welchem Zeitpunkt ausgewertet werden.
\begin{itemize}
    \item Hoch: BitsAboutMe, Datum
    \item Gering: Kaufland Card, Payback, Invisibly
\end{itemize}