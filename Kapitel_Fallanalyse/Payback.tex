\section{Payback} \label{Payback}
Payback GmbH ist die Tochtergesellschaft der Lovalty Partner GmbH, welche wiederum ein Teil der American Express Gruppe ist. \cite{Payback_Info} Payback ist allgegenwärtig in Deutschalnd bekannt, entweder per Kundenkarte oder per App, mit welcher sich Punkte sammeln lassen. \newline
\noindent Payback wurde im Jahre 2000 gegründet und existiert in Deutschland, Italien, Indien, Polen, Mexiko und Österreich. \cite{Payback_Info} \newline

\subsection{Ziel}
In vielen teilnehmenden Einzelhandelsketten wird an der Kassse beim bezahlen nach Payback gefragt. Payback ist dabei ein Truepunkteprogramm (Paybackprogramm) mit dem Kunden bei jedem Einkauf Punkte sammeln können. Dies ist ganz einfach zu erklären, Payback hat über 600 Vertragspartner offline wie online, bei denen mit jedem Einkauf Punkte gesammelt werden können. Dabei kann der Kunde entweder seine Payback Karte vorlegen und diese einscannen lassen oder über das Smartphone die App vorzeigen, um Punkte zu sammeln. \cite{Payback} \newline

\noindent Das Ziel des Karten- bzw. Appinhabers ist dabei viele Punkte zu sammeln, um diese in Coupons und Prämien umtauschen zu können. Womit er für andere Produkte am Ende weniger bezahlen muss. \newline

\noindent Das Ziel von Payback ist es Daten zu generieren, um diese auswerten zu können, z. B. um das Kaufverhalten zu analysieren und dem Kunden dahingehend personalisierte Werbung zu zeigen. Ein weiteres Ziel ist es Verbraucher zu animieren mehr Geld auszugeben, als sie es ohne Rabatte oder Coupons getan hätten. \cite{web}


\subsection{Rollen, Datenfreigabe und -verarbeitung}
Mit der Recherche zu Payback ließen sich drei Rollen heraus kristallisieren, die für das Netzwerk von Payback von Interesse sind. Diese sind im nachfolgendem Fett markiert.\newline
Teilnehmen dürfen nur \textbf{Nutzer} die ihr 16. Lebensjahr vollendet, sowie eine natürliche Person sind und ihren Wohnsitz im europäischen Wirtschaftsraum haben.\cite{Payback_Teilnahme} Das beudeutet Unternehmen sind nicht berechtigt als Nutzer Punkte bei Einkaufen zu sammeln. Möchte der Nutzer nun am Programm teilnehmen, benötigt Payback persönliche Daten, sogenannte Basisdaten. Basisdaten sind:\newline
- Namen\newline
- Geburtsdatum\newline
- Anschrift.\newline
 Ohne diese Daten können zwar Punkte gesammelt werden, allerdings können diese nicht eingelöst werden. Weiterhin besteht die Möglichkeit freiwillig mehr, als die Basisdaten anzugeben. Dies bedeutet der Nutzer fungiert somit als Datenquelle für Payback.

 Payback stellt dabei den \textbf{Verein} dar, welcher alle Daten sammelt, analysiert und aufbereitet. Zum Beispiel verarbeitet es die Basisdaten und die freiwilligen Angaben für die Anmeldung und die Abwicklung. Nebst werden die Richtigkeit und die Vollständigkeit der Adressdaten überprüft, damit postalisch Werbung an den Kunden zugesandt werden kann. Die Daten und ggf. der vom Nutzer vorgenommenen Änderungen der Basisdaten werden auch an die \textbf{Partnerunternehmen} übermittelt, dies gilt nur für das Unternehmen woher der Kunde die Kundenkarte erhalten hat. Eine Ausnahme bildet die Apotheke, dahin werden die Daten nicht übermittelt. Für weitere Übermittlungen an Dritte oder Partnerunternehmen finden nur statt, wenn diesen gesondert vom Nutzer zugestimmt wurden. 

Sobald der Kunde einen Einkauf tätigt, meldet das Partnerunternehmen die Kundennummer und Rabattdaten an Payback.  Ausnahme bildet hier wieder die Apotheke. \newline

\noindent In Absprache mit den Partnerunternehmen stellt Payback Coupons und Prämien zur Verfügung. Diese können von den Nutzern aktiviert werden. Sobald der Nutzer diesen einlöst übermittelt Payack an das Partnerunternehmen, dass diese beim bezahlen berücksichtigt werden.

Weiterhin werden Kommunikationsdaten übermittelt, dies sind alle Angaben, die über das Kundenterminal oder über das Service Center, zur Bearbeitung eines Anliegens des Kunden anfallen. \newline

\noindent Der Nutzer kann entscheiden, ob seine Daten für Zwecke der Marktforschung und Werdbung verwendet werden dürfen. Wenn der Nutzer einwilligt, werden passende werbliche Angebote für Ihn ausgewählt. Diese erfolgen durch die Auswertung der Daten zur Mustererkennung im Einkaufsverhalten. Die Angebote werden entweder postalisch oder per eMail oder per SMS mitgeteilt. Diese Daten werden an Payback und die Partnerunternehmen zum Zwecke für Werbung verarbeitet. Der Kunde hat dabei jederzeit die Möglichkeit die Einwilligung zu widerrufen. Sobald die Einwilligung zur Verarbeitung der Daten für Werbezwecke vorliegt, wird anhand dessen ebenfalls eine Werbeplanung und Erfolgskontrolle durchgeführt. \newline

\noindent Payback speichert die Daten der Kunden, solange diese aktiv am Programm teilnehmen, anschließend werden diese gelöscht. Steuerrechtlich werden diese jedoch 10 Jahre lang aufbewahrt. Die Kommunikationsdaten, zum Beispiel bei Kontaktierung des Service Centers werden nach sptestens 6 Jahren gelöscht.
\cite{Payback_Datenschutz}



\subsection{Daten auf Payback monetarisieren}
Der monetäre Wert für die Nutzerdaten lässt sich nicht in Zahlen bestimmen. 
\noindent Es besteht lediglich die Möglichkeit für den Nutzer die gesammelten Punkte in Coupons und Prämien umzustauschen. Dabei hat der Nutzer die Kontrolle darüber, wie viele Punkte er einlösen möchte. Hierbei sei zu erwähnen, dass bei der normalen Payback Karte die gesammelten Punkte eines jeden Jahres zum 30.09 verfallen, wenn diese nicht verwertet werden. 
Eine weitere Möglichkeit besteht darin, sobald mindestens 200 Punkte erreicht wurden, kann der Nutzer sich diese auf sein Konto auszuahlen lassen. Dabei entsprechen die 200 Punkte einem Betrag von 2€. \cite{Payback_Teilnahme} \newline

\noindent Payback bietet 3 Kartenmodelle für das Punktesammelsystem an. Zum einen die PAyback Karte, welche dauerhaft kostenlos ist, sowie das der Kunde bei allen Vertragspartnern Punke sammeln können. Das zweite Modell ist die Payback American Express Karte, welche ebenfalls kostenlos ist. Hierbei kann der Kunde bei jeder Bezahlung mit dieser Karte Punkte sammeln sei es bei der Bezahlung von Einkäufen bei Partnerunternehmen sowie nicht Partnerunternehmen. Der Vorteil ist bei dieser Karte, dass die Punkte nicht verfallen können und es gibt sogar extra 5000 Payback Punkte bei Kartenabschluss. Die Payback Visa Karte ist das dritte Modell, diese besitzt ebenfalls kein Punkteverfall. Weitere Vorteile dieser Karte sind 0€ Jahresgebühr, 0€ Gebühren bei Bargeldabhebung und auch außerhalb von Partnerunternehmen kann der Kunde bei Bezahlung mit der Karte Punkte generieren. 
Durch diese drei Kartenmodelle hat der Nutzer die Wahl, in welcher Form er am Bonusporgramm teilnehmen möchte und zu welchen Konditionen.
Positiv sei zu erwähnen das Payback einen TÜV geprüften Datenschutz besitzt. \cite{Payback_Karten} 