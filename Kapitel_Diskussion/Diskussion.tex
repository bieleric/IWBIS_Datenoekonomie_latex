\section{Diskussion der Ergebnisse}
Alle betrachteten Fälle der Fallstudie dienen dazu, dass Nutzer ihre Daten monetarisieren können. Auf unterschiedlichen Wegen werden Nutzer dazu motiviert, die analysierten Dienste zu nutzen. Denn nicht nur durch das Angebot der Datenmonetarisierung, sondern auch durch Analyseservices, der Erstellung eines Feeds und personalisierte Werbung versuchen die jeweiligen Dienste ihre Reichweite zu steigern. Es werden demzufolge stets weitere Anreize neben der Monetarisierung geschaffen, das Netzwerk oder den Dienst zu nutzen. Dabei rückt der Verkauf der Daten durch den Nutzer meist in den Hintergrund. So werben BitsAboutMe, Invisibly und Datum beispielsweise damit, wieder die volle Kontrolle über das digitale Leben zu erhalten. Dennoch wird sofort ersichtlich, dass auch Daten aktiv monetarisiert werden können. Kaufland und Payback dagegen werben primär mit Rabattaktionen und Prämien, um die Nutzungsmotivation der Verbraucher zu steigern. Die direkte Monetarisierung der personenbezogenen Daten heben beide Dienste dabei allerdings nicht hervor. Stattdessen geht es aus Sicht der Nutzer darum, Punkte zu sammeln und diese für Prämien oder Rabatte einzulösen. Daher kann angenommen werden, dass die Kenntnis der Nutzer über den Verkauf ihrer Daten limitiert ist und die Aufmerksamkeit von der Datenmonetarisierung abgelenkt wird. \newline

\noindent Darüber hinaus profitieren Payback und Kaufland von einem starken Lock-In-Effekt. Das meint, dass Punkte beispielsweise nicht zwischen anderen Bonusprogrammen übertragbar sind und somit Wechselbarrieren entstehen. Für den Nutzer ist dies weniger wünschenswert, da sie an die entsprechenden Dienste gebunden werden. Bei Kaufland ist dieser Effekt am stärksten zu beobachten, da Punkte und Rabatte lediglich auf dieses Unternehmen beschränkt sind. Payback dagegen bietet mit seinen 600 Online-Shops und 30 Partnern eine größere Auswahl und ermöglicht zudem die Auszahlung der Punkte. Ähnliches gilt für Invisibly, wo die gesammelten Punkte zum jetzigen Stand ausschließlich in bares Geld umgewandelt werden können. Als Gegenbeispiel sind dafür BitsAboutMe und Datum zu erwähnen. Bei BitsAboutMe werden Nutzer direkt mit Geld vergütet und im Datum-Netzwerk erhalten sie DAT-Token, welche auf beliebigen Kryptobörsen eintauschbar sind. \newline 

\noindent Bei der Monetarisierung der Daten spielt der Gegenwert für die Nutzer eine wichtige Rolle. Obwohl überwiegend direktes Geld als Gegenwert für die Daten geboten wird, lässt sich die Fairness des Datenhandels nicht bestimmen. Selbst mit dem im Kapitel \ref{oekonomischerWert} bestimmten Wert personenbezogener Daten lässt sich ein Vergleich nicht durchführen. Dies liegt mitunter an mangelnder Transparenz, sowie an dem Angebot und der Nachfrage auf der jeweiligen Plattform. Da BitsAboutMe, Invisibly und Datum sehr verschiedene Datenmarktplätze sind, lässt sich der Preis nicht konkretisieren. Hinzu kommt, dass die drei Dienste entweder noch im Anfangsstadium ihrer Entwicklung sind oder das Netzwerk noch nicht groß genug ist, um dies abschätzen zu können. Der Grund dafür ist ein indirekter Netzwerkeffekt: Der Wert des Netzwerks steigt durch das steigende Angebot von Nutzerdaten, was mehr Datenkonsumenten anlockt. Diese wachsende Konkurrenz und die steigende Nachfrage führen zu höheren Preisen und somit zu mehr Geld für einzelne Personen. \newline

\noindent Es ist positiv anzumerken, dass Individuen Möglichkeiten haben, ihre persönlichen Daten aktiv zu monetarisieren. Personen werden teilweise jedoch mit Prämien, Rabatten und Geld gelockt, sämtliche Daten über sich preiszugeben. Dabei besteht das Risiko, dass Personen voreilig bzw. unkontrolliert Daten mit Plattformen teilen, die sie danach nicht einfach widerrufen können. Diese Gefahr ist vor allem bei sensiblen persönlichen Daten hoch, beispielsweise bei Bankkonten oder Gesundheitsdaten. Des Weiteren beeinflussen Bonusprogramme das Konsumverhalten einiger Personen, indem sie mit Prämien und Rabatten zu falschen Kaufentscheidungen angeregt werden: Sie kaufen Produkte, die sie ohne Rabatte vielleicht nicht gekauft hätten. Grund dafür ist der Lock-In-Effekt, da Rabatte an bestimmte Produkte gebunden sind und eventuell verfallen, wodurch letztendlich der Gegenwert der monetarisierten Daten verfällt. \newline

\noindent Insgesamt ist die Intention einiger Dienste, Daten kontrolliert zu verkaufen, positiv zu bewerten. Auch der Ansatz persönliche Daten in einem festgelegten Rahmen zu lizenzieren klingt zunächst vielversprechend. Es muss jedoch festgehalten werden, dass die Betreiber der Plattformen den Missbrauch der Daten aus technischen Gründen nicht verhindern können. Denn die in Kapitel \ref{datenoekonomie} beschriebenen Eigenschaften von Daten zeigen, dass Daten beliebig oft ohne Wertverlust vervielfältig werden können. Das bedeutet beispielsweise, dass Datenkonsumenten erworbene Daten auch nach Ablauf der Vertragslaufzeit weiterverwenden oder zu ihren Gunsten weiterverkaufen können. Darüber hinaus werden andere Plattformen, wie beispielsweise Google, nicht davon abgehalten, weiterhin Daten über Individuen zu sammeln und verkaufen. Das bedeutet, dass Personen nicht die volle Kontrolle über ihre persönlichen Daten zurückerlangen, indem sie sich auf der Plattform anmelden. Die Plattformen verhindern nicht die Ausnutzung von Individuen durch Dritte -- vielmehr bieten sie nur einen zusätzlichen Dienst an, mit dem sich Personen aktiv an der Datenökonomie beteiligen können. Obwohl sich der Grad an Aktivität von Plattform zu Plattform etwas unterscheidet, bringen Individuen ihre persönlichen Daten selbst aktiv ein und erhalten dafür einen entsprechenden Gegenwert. Darüber hinaus sei noch erwähnt, dass Personen ihren Profit erhöhen können, indem sie sich bei mehreren Diensten gleichzeitig anmelden und somit ihre Daten mehrfach monetarisieren.
