\section{Der ökonomische Wert von personenbezogenen Daten} \label{oekonomischerWert}

Die Studie \textit{Ökonomischer Wert von Verbraucherdaten für Adress- und Datenhändler}, welche im Auftrag des Bundesministeriums der Justiz und für Verbraucherschutz durchgeführt wurde, ermittelte annäherungsweise den ökonomischen Wert von Nutzerdaten. Sie analysierten dafür die zehn umsatzstärksten Unternehmen Deutschlands im Bereich Datenhandel mit personenbezogenen Daten. Allein 2014 erwirtschafteten diese zehn Unternehmen einen Gesamtumsatz von etwa 450 Mio. Euro, wobei die Schufa beispielsweise im Besitz von 797 Mio. Einzeldaten zu 66,4 Mio. Personen ist. Das bedeutet im Umkehrschluss, dass die Schufa, bei einer Einwohnerzahl von 83,2 Mio. Menschen in Deutschland (Stand 2021 \cite{einwohnerzahl_2021}), Daten von knapp 80\% der deutschen Bevölkerung hat, wobei auf jeden Einzelnen durchschnittlich zwölf Einzeldaten entfallen. Die Studie kommt zu dem Ergebnis, dass der durchschnittliche Wert eines Datensatzes 0,86 Euro beträgt. Ein Datensatz wird dabei aus Post- und E-Mailadressen beziehungsweise Personendaten erstellt, die mit personenbezogenen Merkmalen, wie Konsumverhalten, Bonität, etc., angereichert sind. Bei einer Verwertbarkeit eines Datensatzes von 30 Jahren erzielten die untersuchten Unternehmen damit einen durchschnittlichen Umsatz von 26 Euro pro personenbezogenen Datensatz, wobei dieser je nach Unternehmen zwischen vier und 77 Euro schwankt. \cite{Wert_der_Daten_2017}