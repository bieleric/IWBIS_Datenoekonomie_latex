\section{Diskussion der Ergebnisse}

\subsection{Unterschiede in Kontrolle und Transparenz}
Die Auswertung der Fallstudie zeigt, dass Kontrolle und Transparenz über persönliche Daten zwischen den Diensten stark variieren.

- Kontrolle/Transparenz variiert stark von Dienst zu Dienst
    *Weiterverarbeitung
    *Entfernen von Daten

- Validierung der pers. Daten für Kunsomenten: 
    *BitsAboutMe Abgleich Konto vs Kassenzettel
    *Datum Trust-Ranking-System gut, aber noch offen
- BitsAboutMe und Invisibly sehr ähnlich aber trotzem unterschiedlich
- BAM und Invisibly VS Datum
=> ergebnis man kann sie nicht kategoriesieren, alle sind "einzigartig"


- Motivation für Kunden:
    * BAM Analyseservices
    * Invisibly persönliches Feed, nicht manipulativ
- Lock-In Effekte
    * Bonusprogramme stark
    * Bei Brokern eher schwach
- Anreiz bewerten: Daten preiszugeben für Gegenwert

- Wert der Daten bzw. Wert für Kunden schlecht vergleichbar und schwer zu bestimmen
    * Dienste in unterschiedlichen Stadien (Fertig, Beta, Konzepte)
    * teilweise nicht praktisch testbar
    * unterschiedliche Preismodelle für Datensätze (BAM dynamisch, Invisibly flat)

- Generelle Probleme:
    * Plattformen können Missbrauch nicht verhindern, d.h. Konsumenten können Daten weiterverwenden oder verkaufen => Daten nicht rückrufbar
    * Andere Dienste können weiterhin Daten sammeln und verkaufen (Kontrolle auf Plattform vs. Kontrolle bei Datenquelle)
    * Scope: Händlergebunden vs. Breit Verfügbar

- Erkenntnis:
    * Personen beteiligen sich zwar, aber verkaufen nicht aktiv auf knopfdruck Daten
    * Eher Daten bereitgestellt und warten auf Angebote von Käufern
    * Dennoch aktive Teilnahme an der Datenökonomie