\section{Ergebnisse}
Die in der Fallstudie betrachteten Beispiele zeigen, dass es verschiedene Möglichkeiten für Individuen gibt, ihre persönlichen Daten zu monetarisieren. Während sich die Plattformen in einigen Funktionen unterscheiden, funktionieren sie im Grunde nach demselben Prinzip. Als erstes werden Daten erhoben oder eingebunden, damit sie im zweiten Schritt aufbereitet und weiterverarbeitet werden können. Individuen erhalten hierfür im dritten Schritt einen Gegenwert. Im folgenden Abschnitt werden die Dienste anhand dieser Wertschöpfungskette in drei wesentlichen Punkten verglichen: der Art der Datenerhebung, der Kontrolle und Transparenz über erhobene Daten und abschließend dem Gegenwert, den Individuen für das Teilen ihrer persönlichen Daten erhalten.

\subsection{Datenerhebung}
Datenerhebung meint die Art, wie Dienste an persönliche Daten gelangen bzw. wie Individuen ihre Daten aktiv einbringen können. \newline

\noindent Die beiden Dienste BitsAboutMe und Invisibly unterscheiden sich in diesem Punkt kaum. Auf beiden Plattformen können Benutzer verschiedene Datenquellen, wie beispielsweise Konten aus sozielen Medien oder Bankkonten, einbinden. Das Datum-Projekt verfolgt auch den Ansatz, dass Benutzer externe Dienste als Datenquellen verlinken und so ihre persönlichen Daten teilen. Zum aktuellen Zeitpunkt ist jedoch keine Liste mit möglichen Quellen bekannt. Bei dieser Art der Datenerhebung agieren Individuen lediglich insoweit aktiv, indem sie andere Plattformen verlinken, auf denen bereits personenbezogenen Daten aggregiert worden. \newline

\noindent Auf der anderen Seite steht die Datenerhebung durch das aktive Einbringen neuer persönlicher Daten. Bei BitsAboutMe und Invisibly haben Benutzer die Möglichkeit, eigene Aussagen zur Vervollständigung des Profils zu machen. Personen können Daten aus erster Hand aktiv einbringen, indem sie den eigenen Steckbrief ausfüllen oder mit dem persönlichen Feed interagieren. BitsAboutMe bietet zudem den Dienst an, Kassenzettel einzuscannen. \newline

\noindent Bei den beiden Programmen von Payback und Kaufland können Individuen selbstständig Ihre Daten preisgeben, dabei gibt es Basisdaten die von den Individuen eingegeben werden müssen und freiwillige Angaben. Erst mit der Angabe der Daten können diese am Programm teilnehmen bzw die gesammelten Treuepunkte der Programme in Prämien eintauschen. Weiterhin haben die Individuen die Möglichkeit bei Payback der Datenerhebung für Marketingzwecke zu widersprechen, somit wird das Kaufverhalten nicht analysiert. Kaufland hingegen erhebt diese Daten, um personalisierte Werbung für das Individuum zu schalten. Das Cashback Programm Shoop erhebt keine Daten, um das Kaufverhalten zu analysieren. Lediglich erhebt es die persönlichen Daten, damit eine Auszahlung erfolgen kann.

\subsection{Kontrolle und Transparenz}
Kontrolle bezeichnet den Umfang der aktiven Teilnahme für Individuen beim Benutzen der Plattformen. Dabei ist es wichtig zu unterscheiden, wie groß die Kontrolle einerseits beim Teilen der Daten und andererseits im Nachhinein beim Entfernen ist. Transparenz meint im Kontext der Datenökonomie, in welchem Umfang Benutzer zu einem bestimmten Zeitpunkt über die Verwendung ihrer persönlichen Daten informiert werden. \newline

\noindent Bei den Diensten BitsAboutMe, Datum und Invisibly gilt, dass Benutzer bei der Datenerhebung im Allgemeinen die volle Kontrolle über ihre persönlichen Daten haben. Auf jeder der drei Plattformen ist es Individuen freigestellt, welche externen Datenquellen sie einbinden möchten und welche nicht -- keine Quelle wird erzwungen. Dabei ist es jedoch wichtig zu erwähnen, dass bei der Registrierung auf allen Plattformen die Angabe von Name und E-Mail-Adresse erforderlich ist. Betrachtet man die detailierte Kontrolle beim Einbinden einzelner Datenquellen, so ist festzustellen, dass Personen keine Entscheidungsgewalt haben. Das bedeutet, Benutzer können plattformseitig zunächst nicht einstellen, welche Datensätze konkret aus einer externen Datenquelle übernommen werden sollen. Dies liegt zum Teil daran, dass es sich an dieser Stelle meist um Rohdaten handelt, die erst später zu aussagekräftigen persönlichen Daten aufbereitet werden. \newline

\noindent Bezüglich der Kontrolle und Transparenz über persönliche Daten, die bereits mit den Plattformen geteilt wurden, unterscheiden sich die Fälle stark. \newline

\noindent BitsAboutMe bietet seinen Benutzern eine umfangreiche Kontrolle über die eigenen Daten an. Personen können nachträglich für jede eingebunde Datenquelle entscheiden, welchen konkreten Datensatz sie teilen bzw. löschen wollen. So lassen sich beispielsweise einzelne Beiträge aus dem Instragram-Feed oder abgespielte Lieder von Spotify entfernen. Zudem lassen sich ganze Datenquellen einfach entfernen, wodurch alte Daten automatisch von der Plattform gelöscht werden. Darüber hinaus werden Benutzer stets transparent über die Verwendung ihrer persönlichen Daten aufgeklärt. Bei jedem Angebot teilt die Plattform mit, sowohl welche Profil- als auch Rohdaten mit Dritten geteilt werden und zu welchem Zweck die Daten verwendet werden. Außerdem werden Individuen darüber informiert, wie die Verarbeitung abläuft. Dies beinhaltet insbesondere die Art der Speicherung -- beispielsweise Auswertung auf Servern des Datenkonsumenten -- und die Dauer des Zugriffs. \newline

\noindent Bei Datum stehen dem Nutzer insgesamt fünf Optionen zur Kontrolle über persönliche Daten zur Verfügung. Mit diesen können Personen einschränken, mit welchen Datenkonsumenten die Daten geteilt werden und ob generell eine Gebühr erforderlich ist. Es ist jedoch nicht ersichtlich, ob diese Einstellungen neben einer ganzen Datenquelle auch auf einzelne Datensätze anwendbar sind. Zudem bleibt die Frage offen, inwiefern Individuen ihre persönlichen Daten aus dem verteilten Speicher entfernen können. Mit jedem Datenzugriff durch Dritte werden Personen in einem Transparenzbericht über die Nutzung ihrer Daten informiert. Dabei ist es wichtig zu erwähnen, dass dieser Bericht erst \textit{nach} der Verwendung erstellt wird und somit bereits ein Verkauf stattgefunden hat. \newline

\noindent Obwohl Invisibly mit viel Kontrolle über die eigenen Daten wirbt, stehen Benutzern in der aktuellen Beta-Phase noch wenige Optionen zur Verfügung. Dabei ist es jedoch wichtig zu erwähnen, dass die Plattform nicht direkt getestet werden konnte und alle Informationen somit auf externen Quellen beruhen. Benutzer können geteilte Datenquellen jederzeit entfernen und dadurch den Zugriff für Invisibly widerrufen. Es bleibt aber die Frage offen, was danach mit den bereits durch Invisibly aggregierten Informationen geschieht: Da die Plattform persönliche Daten zur Erstellung eines detaillierten Profils weiterverarbeitet und keine Rohdaten direkt verkauft, können bereits verarbeitete Informationen nicht ohne Weiteres aus dem Profil entfernt werden. Individuen haben allerdings die Möglichkeit, Invisibly direkt zu kontaktieren und um die Löschung von persönlichen Daten zu bitten. Auf diese Weise können sie ebenfalls einen Download ihrer Daten anfordern, welcher als Transparenzbericht über das eigene Profil dient. \newline

\noindent Bei Payback haben die Nutzer nur einen geringen Teil der Kontrolle über ihre Daten. Es gibt Basisdaten die definitiv eingetragen werden müssen und damit verarbeitet werden. Nur über die freiwilligen Angaben haben die Benutzer die Kontrolle, denn diese müssen nicht preisgegeben werden. Weiterhin können die Nutzer entscheiden, ob das Kaufverhalten für Marketingzwecke erhoben werden oder nicht. Falls ja erhält der Nutzer individualisierte Werbung. Allerdings bleibt offen, ob der Nutzer sein Kaufverhalten auch einsehen kann, indem er sieht was er wann, wo gekauft hat. In den AGBs und Datenerhebungen, sowie Datenschutz der Programme ist einzusehen welche Daten erhoben werden und an wen diese übermittelt werden. Bei Payback und Kaufland ist die Übermittlung an Dritte weitestgehend ausgeschlossen, die Daten werden nur unter den Vertragspartnern übermittelt.
Kaufland verarbeitet das Kaufverhalten der Nutzer und wertet diese aus, um dem Nutzer personalisierte Werbung zeigen zu können. Selbstverständilich können Individuen ihre Benutzerkonten schließen. Hierbei unterscheidet Kaufland die Löschung der Kaufland Card und die Löschung des Benutzerkontos, beides muss seperat gelöscht werden. Bestehen noch Punkte auf der Karte so werden diese mit gelöscht. Hingegen bei Payback erfolgt die Löschung der Daten zwar prinzipiell genau, allerdings werden diese für steuerrechtliche Zwecke dennnoch 10 Jahre lang aufgehoben. Bei beiden Programmen können die Individuen jederzeit ihre Daten anpassen, z. B. bei Umzug mit einer neuen Adresse. Falls die Nutzer keine Datenerhebung des Kaufverhalten wünschen, besteht immernoch die Option die Karte beim bezahlen nicht vorzuzeigen.
Beim Cashback-Programm Shoop hat der Nutzer eine hohe Transparenz und Kontrolle über seine Daten, weil zum einen die Daten nicht erhoben werden und zum anderen alle getätigten Einkäufe einsehbar sind.

\subsection{Gegenwert für Individuen}
Auch wenn sich die untersuchten Dienste teilweise deutlich in ihren Funktionen unterscheiden, erhalten Benutzer in jedem Fall einen Gegenwert für ihre aktive Teilnahme an der Datenökonomie. \newline

\noindent Bei den meisten Plattformen handelt es sich dabei um einen monetären Gegenwert. Während Individuen bei BitsAboutMe für jede Transaktion direkt Geld auf ihr Benutzerkonto erhalten, sammeln sie bei Invisibly zunächst Punkte, die sie letztendlich auch in Geld umrechnen und auszahlen können. Auf der Datum-Plattform sammeln Benutzer sogenannte DAT-Token, welche auf Kryptowährungs-Börsen verkauft werden können. \newline

\noindent Dem direkten monetären Erlös stehen alternative Gegenwerte gegenüber, welche auf den einzelnen Plattformen stark variieren. Auf BitsAboutMe erhalten Personen ausführliche Informationen zu ihren persönlichen Daten. Hierfür analysiert der Dienst sämtliche Datensätze aus geteilten Datenquellen und liefert Zusammenfassungen, beispielsweise zum Nutzerverhalten nach der Zeit. Informaionen bekommen Individuen auch auf Invisibly als Gegenwert. Bei dieser Plattform handelt es sich dabei um das persönliche Feed, welches Nutzern für sie relevante Beiträge anzeigt -- basierend auf umfangreichen Analysen der persönlichen Daten. \newline

\noindent Mit der Teilnahme an Programmen, wie Payback, Kaufland und Shoop erhalten Individuen ebenfalls einen Gegenwert, dieser kann sehr stark variieren. Das können zum Beispiel Rabatte, Coupons oder direktes Geld sein. 
Mit der Kaufland Card erhält der Benutzer z. B. gegenüber normalen Einkäufern ohne Card Artikel reduzierter, statt 99ct für Haribo zu zahlen, zahlen diese dann 69ct. Weiterhin sammelt der Nutzer Punkte bei Kaufland, welche dann in Rabatte, Coupons und bei Partnerprogrammen eingetauscht werden können. Die Punkte die bei Payback gesammelt werden, können in Rabatte, Coupons, Prämien und bares Geld umgetauscht werden.
Bei Cashback Programmen, z. B. shoop, erhält das Individuum statt einer Prämie einen direkten Geldwert auf das Konto zurück. 
