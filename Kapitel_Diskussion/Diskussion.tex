\section{Diskussion der Ergebnisse}

\subsection{...}

- Validierung der pers. Daten für Kunsomenten: 
    *BitsAboutMe Abgleich Konto vs Kassenzettel
    *Datum Trust-Ranking-System gut, aber noch offen
- BitsAboutMe und Invisibly sehr ähnlich aber trotzem unterschiedlich
- BAM und Invisibly VS Datum
- Basisdaten werden bei Bonusprogrammen verarbeitet, wer freiwillige Angaben angibt, so werden diese auch verarbeitet (Kontrolle erfolgt über Bestätigungslinks oder durch Zusendung von Werbung auf dem postalischen Wege)

- Motivation für Kunden:
    * BAM Analyseservices
    * Invisibly persönliches Feed, nicht manipulativ
    * Coupons, Rabatte, bares Geld (Anreiz)
- Lock-In Effekte
    * Bonusprogramme stark
    * Bei Brokern eher schwach
- Anreiz bewerten: Daten preiszugeben für Gegenwert

- Wert der Daten bzw. Wert für Kunden schlecht vergleichbar und schwer zu bestimmen
    * Dienste in unterschiedlichen Stadien (Fertig, Beta, Konzepte)
    * teilweise nicht praktisch testbar
    * unterschiedliche Preismodelle für Datensätze (BAM dynamisch, Invisibly flat)
    *bei Bonusprogrammen erzielt man mehr durch das Kaufverhalten Werte, als durch persönliche Angaben

- Generelle Probleme:
    * Plattformen können Missbrauch nicht verhindern, d.h. Konsumenten können Daten weiterverwenden oder verkaufen => Daten nicht rückrufbar
    * Andere Dienste können weiterhin Daten sammeln und verkaufen (Kontrolle auf Plattform vs. Kontrolle bei Datenquelle)
    * Scope: Händlergebunden vs. Breit Verfügbar
    * Daten werden lange aufgehoben

- Erkenntnis:
    * Personen beteiligen sich zwar, aber verkaufen nicht aktiv auf knopfdruck Daten
    * Eher Daten bereitgestellt und warten auf Angebote von Käufern
    * Dennoch aktive Teilnahme an der Datenökonomie, durch evtl. Zustimmung des Nutzers zu Marketingzwecken und damit sein Kaufverhalten
    * Personen werden dazu animiert die Daten zu verkaufen, um damit etwas Geld zu verdienen
    * zudem werden Personen dazu animiert durch Rabatte und Coupons etwas zu kaufen, was sie ohne diese nicht tun würden.
