\section{Diskussion der Ergebnisse}

\subsection{Kategorisierung der Dienste}
Die Ergebnisse der Fallstudie deuten darauf hin, dass zwischen den Plattformen Gemeinsamkeiten bestehen, mit denen sich Kategorien bilden lassen. Einerseits können die Dienste nach der \textit{Art der Daten} unterteilt werden, welche Individuen monetarisieren. Andererseits lassen sich die Plattformen nach dem \textit{Gegenwert} kategoriesieren, den Personen für die Bereitstellung persönlicher Daten erhalten. Abschließend unterscheiden sich die Dienste auch in ihrer Funktionsweise. \newline

\noindent \textbf{Unterscheidung nach Art der Daten:} Bei dieser Kategorisierung wird zwischen den Daten unterschieden, die Individuen aktiv mit der Plattform teilen. Die Ergebnisse zeigen, dass es hier zwei Kategorien gibt. Die erste Gruppe lässt sich unter dem Begriff \textit{Konsumdaten} zusammenfassen und beinhaltet die beiden Anbieter Payback und Kaufland Card. Der Fokus dieser Dienste liegt darauf, dass Individuen Daten zu ihrem Konsumverhalten zur Datenanalyse freigeben und dafür entlohnt werden. Die Dienste Invisibly und BitsAboutMe lassen sich hier auch teilweise zuordnen, da bei ihnen Bankkonten hinterlegt bzw. Kassenzettel hochgeladen werden können. In der zweiten Gruppe befinden sich Dienste, auf denen Personen sämtliche \textit{Daten zum persönlichen Profil} monetarisieren können. Als Datenquellen dienen hier insbesondere Social Media Accounts und eigene Angaben zur Person. In diese Gruppe fallen die Anbieter BitsAboutMe, Datum und Invisibly. \newline

\noindent \textbf{Unterscheidung nach Gegenwert:} Beim Gegenwert bilden sich drei verschiedene Gruppen zur Kategorisierung. Die erste Gruppe umfasst alle Dienste, bei denen Individuen \textit{bares Geld} für ihre persönlichen Daten erhalten. Darunter fallen die Anbieter BitsAboutMe, Datum und Invisibly sowie Payback durch seine Auszahlungsfunktion. Die zweite Gruppe beinhaltet alle Dienste, auf denen Personen einen \textit{geldwerten Gegenwert} erhalten. Damit sind Gutscheine, Rabatte und Prämien gemeint, da diese sich insofern von Bardgeld unterscheiden, dass sie beim Einlösen an Bedingungen wie beispielsweise bestimmte Partner gebunden sind. Zu dieser Gruppe zählen Payback und Kaufland Card. Die letzte Gruppe beinhaltet alle Anbieter, bei denen Individuen \textit{Informationen/Wissen} für ihre persönlichen Daten erhalten. Darunter fallen BitsAboutMe mit den Analyseservices, Invisibly mit dem persönlichen Feed für Benutzer sowie Payback und Kaufland Card mit personalisierter Werbung. \newline



\noindent \textbf{Unterscheidung nach Funktionsweise:} Die Recherche zu den Anbietern hat gezeigt, dass Dienste zur Monetarisierung von Konsumdaten breits unter dem Begriff \textit{Bonusprogramme} zusammengefasst werden. Diese unterteilen sich in die beiden Gruppen Payback Programme, bei denen Personen Gutscheine und Rabatte erhalten, und Cashback Programmen, bei denen Personen bares Geld ausgezahlt bekommen. Payback Programme meint in diesem Kontext sämtliche Anbieter und nicht das Unternehmen Payback selbst. Zu den Bonusprogrammen zählen die betrachteten Plattformen Payback und Kaufland Card. Darüber lassen sich Dienste zusammenfassen, die als \textit{Marktplatz} agieren. Auf ihnen werden Individuen als Datenanbieter mit Werbetreibenden und anderen Dritten als Datenkonsumenten in Verbindung gebracht, wobei Daten entweder verkauft oder zur Nutzung lizenziert werden. Zu den Marktplätzen zählen BitsAboutMe, Datum und Invisibly. \newline

\subsection{Unterschiede in Kontrolle und Transparenz}
Die Auswertung der Fallstudie zeigt, dass Kontrolle und Transparenz über persönliche Daten und freiwillige Daten zwischen den Diensten stark variieren. Weiterhin wird das Nutzerverhalten und das Kaufverhalten ausgewertet.

- Kontrolle/Transparenz variiert stark von Dienst zu Dienst
    *Weiterverarbeitung
    *Entfernen von Daten
    *zum Teil kann man der Aufzeichnung des Nutzerverhalten/ Kaufverhalten von Personen für die Marketingzwecken widersprechen

- Validierung der pers. Daten für Kunsomenten: 
    *BitsAboutMe Abgleich Konto vs Kassenzettel
    *Datum Trust-Ranking-System gut, aber noch offen
- BitsAboutMe und Invisibly sehr ähnlich aber trotzem unterschiedlich
- BAM und Invisibly VS Datum
- Basisdaten werden bei Bonusprogrammen verarbeitet, wer freiwillige Angaben angibt, so werden diese auch verarbeitet (Kontrolle erfolgt über Bestätigungslinks oder durch Zusendung von Werbung auf dem postalischen Wege)



- Motivation für Kunden:
    * BAM Analyseservices
    * Invisibly persönliches Feed, nicht manipulativ
    * Coupons, Rabatte, bares Geld (Anreiz)
- Lock-In Effekte
    * Bonusprogramme stark
    * Bei Brokern eher schwach
- Anreiz bewerten: Daten preiszugeben für Gegenwert

- Wert der Daten bzw. Wert für Kunden schlecht vergleichbar und schwer zu bestimmen
    * Dienste in unterschiedlichen Stadien (Fertig, Beta, Konzepte)
    * teilweise nicht praktisch testbar
    * unterschiedliche Preismodelle für Datensätze (BAM dynamisch, Invisibly flat)
    *bei Bonusprogrammen erzielt man mehr durch das Kaufverhalten Werte, als durch persönliche Angaben

- Generelle Probleme:
    * Plattformen können Missbrauch nicht verhindern, d.h. Konsumenten können Daten weiterverwenden oder verkaufen => Daten nicht rückrufbar
    * Andere Dienste können weiterhin Daten sammeln und verkaufen (Kontrolle auf Plattform vs. Kontrolle bei Datenquelle)
    * Scope: Händlergebunden vs. Breit Verfügbar
    * Daten werden lange aufgehoben

- Erkenntnis:
    * Personen beteiligen sich zwar, aber verkaufen nicht aktiv auf knopfdruck Daten
    * Eher Daten bereitgestellt und warten auf Angebote von Käufern
    * Dennoch aktive Teilnahme an der Datenökonomie, durch evtl. Zustimmung des Nutzers zu Marketingzwecken und damit sein Kaufverhalten
    * Personen werden dazu animiert die Daten zu verkaufen, um damit etwas Geld zu verdienen
    * zudem werden Personen dazu animiert durch Rabatte und Coupons etwas zu kaufen, was sie ohne diese nicht tun würden.
