\begingroup
\begin{table}[!ht]
\renewcommand{\arraystretch}{1.5} % vertical padding
\begin{tabularx}{\textwidth}{| X | X |}
\hline
\rowcolor[rgb]{0.737,0.839,0.862}
\textbf{Kategorie} & \textbf{Personenbezogene Daten} \\
\hline
Soziodemografische/-ökonomische Angaben & Alter, Geschlecht, Bildung, Beruf, Einkommen, Familienstand etc. \\
\hline
Geografische Angaben & Standort, Anschrift etc. \\
\hline
Sensible Daten & Ethnische  Herkunft,  politische  Meinungen, Religions-
zugehörigkeit,  Gesundheitsdaten,  Sexualleben,  bio-
metrische Daten etc. \\ 
\hline
Persönlichkeitsprofil & Extraversion, Gewissenhaftigkeit etc. \\
\hline
Angaben über Konsumverhalten & Getätigte Einkäufe etc. \\
\hline
Interessen & Produkte, Marken, Musik, Film etc. \\
\hline
Technische Angaben & Browser, Endgerät, IP-Adressen, Nutzungs-/Surfverhalten (Web-Tracking), Cookies etc. \\
\hline
Werturteile & Schul-, Studienabschluss-, Arbeitszeugnisse, Diplome, Zertifikate, etc. \\
\hline
Audiovisuelle Daten & Videoaufzeichnungen, Fotos, Audio-Mitschnitte etc. \\
\hline
Biometrische Daten & Geschlecht, Haut-, Haar-, Augenfarbe, Statur, Kleidergröße, etc. \\
\hline
Indirekte bzw. personenbeziehbare Daten & Personalausweisnummer, Versicherungsnummer, Kreditkartennummer, Kfz-Kennzeichen, Telefonnummer, E-Mail-Adresse etc. \\
\hline
\end{tabularx}
\caption{Arten von personenbezogenen Daten (Angelehnt an: \textit{Ökonomischer Wert von Verbraucherdaten für Adress- und Datenhändler S.7} \cite{Wert_der_Daten_2017})}
\label{tab:personenbezogeneDaten}
\end{table}
\endgroup


