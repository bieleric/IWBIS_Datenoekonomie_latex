\section{Diskussion der Ergebnisse}
Alle betrachteten Fälle der Fallstudie dienen dazu, dass Nutzer ihre Daten monetarisieren können. Auf unterschiedlichen Wegen werden Nutzer dazu motiviert, die analysierten Dienste zu nutzen. Denn nicht nur durch das Angebot der Datenmonetarisierung, sondern auch durch Analyseservices, der Erstellung eines Feeds und personalisierte Werbung versuchen die jeweiligen Dienste ihre Reichweite zu steigern. Es werden demzufolge stets weitere Anreize neben der Monetarisierung geschaffen, das Netzwerk oder den Dienst zu nutzen. Dabei rückt der Verkauf der Daten durch den Nutzer meist in den Hintergrund. So werben BitsAboutMe, Invisibly und Datum beispielsweise damit, volle Kontrolle über das digitale Leben zurückzuerhalten. Dennoch wird sofort ersichtlich, dass auch Daten aktiv monetarisiert werden können. Kaufland und Payback dagegen werben primär mit Rabattaktionen und Prämien, um die Nutzungsmotivation der Verbraucher zu steigern. Die direkte Monetarisierung der personenbezogenen Daten heben beide Dienste dabei allerdings nicht hervor. Stattdessen geht es aus Sicht der Nutzer darum, Punkte zu sammeln und diese für Prämien oder Rabatte einzulösen. Daher kann angenommen werden, dass die Kenntnis der Nutzer über den Verkauf ihrer Daten limitiert ist und die Aufmerksamkeit von der Datenmonetarisierung abgelenkt wird. \newline

\noindent Darüber hinaus profitieren Payback und Kaufland von einem starken Lock-In-Effekt. Das meint, dass Punkte beispielsweise nicht zwischen anderen Bonusprogrammen übertragbar sind und somit Wechselbarrieren entstehen. Für den Nutzer ist dies weniger wünschenswert, da sie an die entsprechenden Dienste gebunden werden. Bei Kaufland ist dieser Effekt am stärksten zu beobachten, da Punkte und Rabatte lediglich auf dieses Unternehmen beschränkt sind. Payback dagegen bietet mit seinen 600 Online-Shops und 30 Partnern eine größere Auswahl und ermöglicht zudem die Auszahlung der Punkte. Ähnliches gilt für Invisibly, wo die gesammelten Punkte zum jetzigen Stand ausschließlich in bares Geld umgewandelt werden können. Es ist jedoch absehbar, dass auch hier eine Lock-In-Strategie verfolgt wird. Als Gegenbeispiel sind dafür BitsAboutMe und Datum zu erwähnen. Bei BitsAboutMe werden Nutzer direkt mit Geld vergütet und im Datum-Netzwerk erhalten sie DAT-Token, welche auf beliebigen Kryptobörsen eintauschbar sind. \newline 

\noindent Bei der Monetarisierung der Daten spielt der Gegenwert für die Nutzer eine wichtige Rolle. Obwohl überwiegend direktes Geld als Gegenwert für die Daten geboten wird, lässt sich die Fairness des Datenhandels nicht bestimmen. Selbst mit dem im Kapitel \ref{oekonomischerWert} bestimmten Wert personenbezogener Daten lässt sich ein Vergleich nicht durchführen. Dies liegt mitunter an mangelnder Transparenz, sowie an dem Angebot und der Nachfrage auf der jeweiligen Plattform. Da BitsAboutMe, Invisibly und Datum sehr verschiedene Datenmarktplätze sind, lässt sich der Preis nicht konkretisieren. Hinzu kommt, dass die drei Dienste entweder noch im Anfangsstadium ihrer Entwicklung sind oder das Netzwerk noch nicht groß genug ist, um dies abschätzen zu können. Dem zugrunde liegt ein indirekter Netzwerkeffekt: Der Wert des Netzwerks steigt durch das steigende Angebot von Nutzerdaten, was mehr Datenkonsumenten anlockt. Diese wachsende Konkurrenz und die steigende Nachfrage führen zu höheren Preisen und somit zu mehr Geld für die Individuen. \newline

\noindent Insgesamt ist die Intention einiger Dienste, die Daten kontrolliert zu verkaufen, positiv zu bewerten. Auch der Ansatz persönliche Daten in einem festgelegten Rahmen zu lizenzieren klingt zunächst vielversprechend. Es muss jedoch auch festgehalten werden, dass die Betreiber der Plattformen den Missbrauch der Daten aus technischen Gründen nicht verhindern können. Denn die in Kapitel \ref{datenoekonomie} beschriebenen Eigenschaften von Daten zeigen, dass Daten beliebig oft ohne Wertverlust vervielfältig werden können.


\subsection{...}

- Generelle Probleme:
    * Plattformen können Missbrauch nicht verhindern, d.h. Konsumenten können Daten weiterverwenden oder verkaufen => Daten nicht rückrufbar
    * Andere Dienste können weiterhin Daten sammeln und verkaufen (Kontrolle auf Plattform vs. Kontrolle bei Datenquelle)
    * Scope: Händlergebunden vs. Breit Verfügbar
    * Daten werden lange aufgehoben


- Validierung der pers. Daten für Kunsomenten: 
    *BitsAboutMe Abgleich Konto vs Kassenzettel
    *Datum Trust-Ranking-System gut, aber noch offen
- BitsAboutMe und Invisibly sehr ähnlich aber trotzem unterschiedlich
- BAM und Invisibly VS Datum
- Basisdaten werden bei Bonusprogrammen verarbeitet, wer freiwillige Angaben angibt, so werden diese auch verarbeitet (Kontrolle erfolgt über Bestätigungslinks oder durch Zusendung von Werbung auf dem postalischen Wege)


- Erkenntnis:
    * Personen beteiligen sich zwar, aber verkaufen nicht aktiv auf knopfdruck Daten
    * Eher Daten bereitgestellt und warten auf Angebote von Käufern
    * Dennoch aktive Teilnahme an der Datenökonomie, durch evtl. Zustimmung des Nutzers zu Marketingzwecken und damit sein Kaufverhalten
    * Personen werden dazu animiert die Daten zu verkaufen, um damit etwas Geld zu verdienen
    * zudem werden Personen dazu animiert durch Rabatte und Coupons etwas zu kaufen, was sie ohne diese nicht tun würden.
    * alle können gleichzeitgi verwendet werden



==============
- Motivation für Kunden:
    * BAM Analyseservices
    * Invisibly persönliches Feed, nicht manipulativ
    * Coupons, Rabatte, bares Geld (Anreiz)
- Lock-In Effekte
    * Bonusprogramme stark
    * Bei Brokern eher schwach
- Anreiz bewerten: Daten preiszugeben für Gegenwert

- Wert der Daten bzw. Wert für Kunden schlecht vergleichbar und schwer zu bestimmen
    * Dienste in unterschiedlichen Stadien (Fertig, Beta, Konzepte)
    * teilweise nicht praktisch testbar
    * unterschiedliche Preismodelle für Datensätze (BAM dynamisch, Invisibly flat)
    *bei Bonusprogrammen erzielt man mehr durch das Kaufverhalten Werte, als durch persönliche Angaben