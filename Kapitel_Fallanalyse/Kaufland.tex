\section{Kaufland Card}
Die Kaufland Card ist eine Vorteilskarte für die Einzelhandelskette Kaufland. Die Kaufland Card existiert seit dem 28.10.2021 in Deutschland, wurde aber zuvor schon 2019 in Osteuropa eingeführt.

\subsection{Rollen und Ziele}
Kaufland verfolgt mit der Einführung der Kaufland-App das Ziel, den Einkaufsprozess zu digitalisieren und somit zu vereinfachen. Darunter zählen beispielsweise digitale Preisschilder, digitale Kassenbons, kontaktloses Bezahlen aber auch die Kaufland Card. \cite{Kaufland_Ziele} Mit der Kaufland Card sollen Kunden an das Unternehmen gebunden und das Kaufverhalten der Nutzer verstanden werden, um somit relevante und personalisierte Werbung zu präsentieren. \cite{Kaufland_Datenschutz} Das bedeutet, es gibt folgende zwei Rollen: \newline

\noindent \textbf{Nutzer:} Der Nutzer kann sich einerseits eine digitale Kaufland Card in der Kaufland-App oder andererseits eine klassische Kaufland Card in einer Filiale verschaffen, um somit am Bonusprogramm teilnehmen zu können. Damit hat er die Möglichkeit Treuepunkte für einen Einkauf zu bekommen, Sparangebote und Coupons zu erhalten, sowie an Gewinnspielen teilnehmen zu können. Für die Registrierung des Einkaufs und die Erfassung der Treuepunkte ist es dabei notwendig, den zur Kaufland Card zugehörigen Barcode bei dem Abschluss eines Einkaufs vorzuzeigen. \newline

\noindent \textbf{Kaufland:} Kaufland fungiert bei diesem Bonusprogramm einerseits als Datenkonsument, welcher die Daten der Nutzer sammelt und verarbeitet aber andererseits auch als Serviceanbieter, welcher die gesammelten Daten analysiert und die Inhalte auf den Nutzer zuschneidet. Damit will Kaufland den Einkaufsprozess vereinfachen und die eigene Angebote bewerben und hervorheben.


\subsection{Datenerhebung}
Bei der Registrierung für die Kaufland Card benötigt Kaufland folgende Daten: E-Mail Adresse oder Mobilfunknummer, Name, Geschlecht, Geburtsdatum und Adresse. Darüber hinaus werden die Einkaufsdaten bei dem Vorzeigen des Barcodes erhoben. Das bedeutet, dass der Nutzer selbst die Wahl hat, ob sein Einkauf registriert wird und damit seine Einkaufsdaten erhoben werden. Entscheidet sich dieser jedoch gegen das Vorzeigen des Barcodes, wird damit auch die Teilnahme am Bonusprogramm verwehrt. Für den Nutzer besteht jedoch die Möglichkeit die registrierten Einkäufe und Kassenbons in der Kaufland-App einzusehen -- nicht jedoch sie zu bearbeiten oder zu löschen. Dies ist nur durch die Löschung des gesamten Kundenkontos möglich. \cite{Kaufland_FAQ} Bezüglich des Datenschutzes und der zusammengehörigen Datenverarbeitung garantiert Kaufland sich an die geltenden Datenschutzbestimmungen zu halten und sie nicht an Dritte weiter zu geben. \cite{Kaufland_Rechtliches} 

\subsection{Daten bei Kaufland monetarisieren}
Mit der Kaufland Card in der dazugehörigen Erhebung der Einkaufsdaten wird dem Nutzer ein Gegenwert geboten. Dieser umfasst zum einen einen exklusiven Preisnachlass für ausgewählte Artikel und zum anderen Prämien, die durch den Eintausch gesammelter Treuepunkte erhältlich sind. Einen Treuepunkt erhält man pro fünf Euro Einkaufswert, welcher nach einem Jahr wiederum verfällt. Eine Umwandlung der Punkte in Geld ist allerdings mit der Kaufland-App nicht möglich. \cite{Kaufland_FAQ} Zusammengefasst bedeutet das, dass folgende Schritte für die Monetariserung der Daten notwendig sind:

\begin{enumerate}
	\item Bei Kaufland registieren (App oder Filiale) 
	\item Personenbezogene Daten hinterlegen
	\item Barcode bei einem Einkauf vorzeigen
	\item Punkte einlösen
\end{enumerate}
