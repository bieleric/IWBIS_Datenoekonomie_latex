\section{Ziel und Methodik}
Aus diesem Grund beschäftigt sich diese Arbeit mit den verschiedenen Möglichkeiten für Personen, sich aktiv in der Datenökonomie zu beteiligen. In dieser Arbeit wird dargestellt, bei welchen Diensten individuelle Personen ihre Daten anderen Akteuren der Datenökonomie zur Verfügung stellen können und welchen Gegenwert sie dafür jeweils erhalten. Der Fokus liegt dabei ausschließlich auf persönliche Daten, da diese sehr empfindliche Informationen über Individuen enthalten und deshalb für Unternehmen von großem Interesse sind. Zudem betrachten wir für jeden Fall, mit welcher Kontrolle und Transparenz die Daten geteilt werden. Daher lautet die Forschungsfrage dieser Arbeit:

\begin{center}
\textit{Welche Möglichkeiten gibt es für Personen, aktiv ihre persönlichen Daten ökonomisch zu verwerten?}
\end{center}

\noindent Um diese Frage zu beantworten beschäfti sich diese Arbeit zunächst mit den Begriffen \textit{personenbezogene Daten} und \textit{Datenökonmie}. Die Definition der Datenökonomie mit einer Abgrenzung zu anderen wirtschaftlichen Bereichen bildet die Voraussetzung für die darauffolgende Fallstudie. In dieser wurden sieben verschiedene Anbieter zur monetarisierung persönlicher Daten verglichen. Obwohl die ausgewählten Unternehmen vielfältige Ziele verfolgen und sehr unterschiedlich funktionieren, lassen sie sich der Datenökonomie zuordnen. Nach Auswahl einiger Fallbeispiele wurde jeder Dienst einzeln analysiert. Dabei wird zunächst auf die Ziele der Anbieter eingegangen und anschließend wird nachvollziehbar dargestellt, wie Individuen der Datenökonomie ihre persönlichen Daten auf der jeweiligen Plattform monetarisieren können. Zusätzlich wird in jedem Fallbeispiel beschrieben, welches Maß an Kontrolle und Transparenz den Benutzern offenbart wird. \newline

\noindent Der darauffolgenden Abschnitt beschäftigt sich mit der Auswertung der Ergebnisse der Fallstudie. Hierfür werden die Erkenntnisse zu den einzelnen Anbietern miteinander in Verbindung gebracht, um Gemeinsamkeiten und Unterschiede darzustellen. Im Mittelpunkt stehen dabei die Aspekte der Datenerhebung, Kontrolle und Transparenz über die Daten, sowie der Gegenwert für Individuen. Dies ermöglicht abschließend, einige Themen -- wie beispielsweise die Ausnutzung von Personen -- kritisch zu hinterfragen und allgemeine Probleme abzuleiten. 