\documentclass[
 openright,
 %twoside, % beidseitig für Druck
 a4paper
]{scrreprt}
%scrreprt
\input{htwcd/htwcd_content.sty}

% Vorgaben zum Seitenrand	
\geometry{
	top=30mm, 
	bottom=30mm, 
	%headsep=15mm, 
	inner=40mm, 
	outer=20mm, 
	%left=40mm,
	%right=20mm,
	%footskip=15mm,
}

\usepackage{scrlayer-scrpage}
\pagestyle{scrheadings}
\clearpairofpagestyles
\ofoot{\pagemark}
\raggedbottom

% Codierung
\usepackage[utf8]{inputenc}
\usepackage[T1]{fontenc}
\usepackage[ngerman]{babel}
\usepackage{csquotes}	% Anführungszeichen
\usepackage{amssymb}

% Grafiken
\usepackage{graphicx}
\graphicspath{ {./images/} }
\usepackage{float}
\usepackage{longtable}

% Quelltext
\usepackage{listings}
\usepackage{scrhack}

% Links
\usepackage{xurl}
\usepackage[bookmarks,%
bookmarksopen=false,% Klappt die Bookmarks in Acrobat aus
colorlinks=true,%
linkcolor=black,%
citecolor=black,%
urlcolor=black,%
]{hyperref}

\usepackage{datetime}


\usepackage{tikz}
\usetikzlibrary{positioning,shadings}
\usetikzlibrary{arrows}

% ======================================================
% Informationen für das Dokument und Titelseite:
% ======================================================
\faculty{Fakultät Informatik/Mathematik}

\title{Persönliche Daten in der Datenökonomie}

\author{
	Maria Mukian\\
	Philipp Steigler\\
	Eric Hans Gero Biele
}

\professor{Prof. Dr. Jürgen Anke}

\newdate{abgabe}{14}{01}{2021}
\date{\displaydate{abgabe}}
% ======================================================


% ======================================================
% Literatur/Quellen und Akronyme/Glossar:
% ======================================================
\usepackage[
backend=bibtex,
sortlocale=de_DE, 
bibencoding=utf8,
style=numeric, 
citestyle=numeric,
doi=false,
isbn=false,
url=false,
sorting=nty
]{biblatex}	% BibTeX
\bibliography{Bibliography}

\usepackage[acronym]{glossaries}
\makeglossaries
% ------------------------------------------
% Glossar  
% keine Umlaute im entry verwenden   
\newglossaryentry{natuerlichePersonG}
{
name={natürliche Person},
description={\glqq Eine natürliche Person meint den Menschen als Rechtssubjekt und somit als Träger von Rechten und Pflichten.\grqq{} \cite{NatPerson_2018}}
}

%============================================
% Akronyme
%============================================
\newglossaryentry{DSGVO}{
	type=\acronymtype, 
	name={DSGVO}, 
	description={Datenschutz-Grundverordnung}
}
% ======================================================
\pagenumbering{roman}

\begin{document}
% ==============================================
% Deckblatt
% ==============================================
\maketitle

% ==============================================
% Abstract
% ==============================================
\setcounter{page}{1}
\pagenumbering{gobble}
\thispagestyle{empty}
\vspace*{5cm}
\begin{center}
\begin{minipage}{0.9\textwidth}
\chapter*{Abstract}

Die vorliegende Arbeit gibt einen Überblick über die Möglichkeiten der aktiven Monetarisierung von persönlichen Daten mit deren Potenzial und Einschränkungen. Im Zuge dessen wurden die Grundlagen der Datenökonomie, sowie der monetäre Wert von Daten untersucht und darauf aufbauend eine Fallstudie durchgeführt, welche die Basis dieser Arbeit bildet. Die Ergebnisse der Fallstudie führten zu einer Kategorisierung der Monetarisierungsmöglichkeiten durch den Verbraucher. Diese Arbeit ist sowohl für IT- und Wirtschaftswissenschaftler, als auch für Unternehmen und Verbraucher selbst interessant, die sich auf datengetriebene Technologien stützen.

\vspace*{1.5cm}
\end{minipage}
\end{center}
\clearpage

% ==============================================
% Verzeichnisse:
% ==============================================
% Inhaltsverzeichnis
\pagenumbering{roman}
\setcounter{page}{1}
\tableofcontents
% Abkürzungsverzeichnis
% ACHTUNG: Akronyme und Glossar muss (ähnlich wie bibtex) extra kompiliert werden über makeglossaries (in texmaker bspw. "makeglossaries %")
\printglossary[type=\acronymtype]
\addcontentsline{toc}{chapter}{Akronyme}
% Glossar
\printglossary[type=main]
\addcontentsline{toc}{chapter}{Glossar}
% Abbildungsverzeichnis
\listoffigures
\addcontentsline{toc}{chapter}{Abbildungsverzeichnis}
% Tabellenverzeichnis
\listoftables
\addcontentsline{toc}{chapter}{Tabellenverzeichnis}

% ==============================================
% Textteil
% Richtlinie: 25-40 Seiten
% \section{}, \subsection{}, \subsubsection{}
% ==============================================
\clearpage
\pagenumbering{arabic}
\chapter{Einleitung}
% * Einleitung
\section{Motivation}
Die Verbreitung des Internets auf Computern und mobilen Endgeräten hat zu einer Produktion riesiger Datenmengen geführt. Diese Daten -- insbesondere personenbezogene Daten -- haben sich zu einer wichtigen Ressource entwickelt, mit der Unternehmen einen Mehrwert erzeugen. Im Weiteren entstand daraus eine Datenökonomie, die allein auf dem Handel und der Verarbeitung personenbezogener Daten beruhen. \cite{humanDemand_2020} Dabei werden Daten meist jedoch nicht direkt von den Erzeugern, den individuellen Personen, verkauft. Stattdessen erheben große Internetdienste Anspruch auf die Daten ihrer Nutzer durch Klauseln in den Nutzungsbedingungen und verkaufen diese dann an Dritte weiter -- oftmals ohne Nutzer im Einzelfall darüber in Kenntnis zu setzen. Obwohl Individuen durch Erzeugung von Daten die Grundlage solcher Ökosysteme bilden, werden sie von Unternehmen bislang nur als Datenquelle gesehen und somit ausgenutzt. \cite{monetizingData_2016} \newline

\noindent Viele Personen werden sich in letzer Zeit allerdings zunehmend bewusst, dass Technologieunternehmen diese Daten sammeln und verkaufen. Darüber hinaus wird ihnen deshalb bewusst, dass ihre persönlichen Daten einen gewissen Wert haben und dass sie mehr Kontrolle über deren Verwendung haben sollten. Rantanen und Koskinen fanden in einer Studie heraus, dass Personen zwei wesentliche Forderungen an eine faire Datenökonomie stellen: transparente Kommunikation und eine aktive Rolle im Handel mit den eigenen Daten. Eine aktive Rolle in der Datenökonomie meint dabei, dass Individuen die Nutzung ihrer eigenen Daten kontrollieren und für die Teilnahme einen entsprechenden Gegenwert erhalten wollen. \cite{humanDemand_2020} \newline

\noindent Bereits im Jahr 2016 stellte Batineh et. al jedoch fest, dass es an Plattformen mangelt, auf denen Individuen als primäre Datenerzeuger ihre Daten für einen entsprechenden Gegenwert selbst anbieten können. \cite{monetizingData_2016}
% * Ziel und Methodik
\section{Ziel und Methodik}

Aus diesem Grund beschäftigen wir uns im Folgenden mit den verschiedenen Möglichkeiten für Personen, sich aktiv in der Datenökonomie zu beteiligen. In dieser Arbeit stellen wir dar, bei welchen Diensten individuelle Personen ihre Daten anderen Akteuren der Datenökonomie zur Verfügung stellen können und welchen Gegenwert sie dafür jeweils erhalten. Wir fokussieren uns dabei ausschließlich auf persönliche Daten, da diese sehr empfindliche Informationen über Individuen enthalten und deshalb für Unternehmen von großem Interesse sind. Zudem betrachten wir für jeden Fall, mit welcher Kontrolle und Transparenz die Daten geteilt werden. Unsere Forschungsfrage lautet deshalb:
\\
\\
\centerline{\textit{Welche Möglichkeiten gibt es für Personen, aktiv ihre Daten ökonomisch zu verwerten?}}
\\
\\
Um diese Frage zu beantworten haben wir ...

\chapter{Grundlagen}
% * Datenschutz
\section{Datenschutz}
\subsection{Personenbezogene Daten}
Laut der Definition in der Datenschutz Grundverordnung (\gls{DSGVO}) im Kapitel 1 Artikel 4 Nummer 1 werden personenbezogene Daten als diejenigen Informationen bezeichnet, mit denen sich \gls{natuerlichePersonG}en identifizieren lassen. Die Identifikation kann auf direktem oder indirektem Wege erfolgen, insbesondere durch die Zuordnung eines Namen zu Kennnummern, Standortdaten oder anderen psychischen, physiologischen, genetischen, wirtschaftlichen, kulturellen oder sozialen Merkmalen der natürlichen Person. \cite{DSGVO_Art4} Gemäß der europäischen Union werden Teilinformationen, die zusammen zur Identifizierung einer natürlichen Person dienen, ebenfalls als personenbezogene Daten kategorisiert. Werden diese Daten anonymisiert und lassen keinen Schluss auf die natürliche Person zu, so werden diese Daten nicht mehr als personenbezogen betrachtet. Beispiele für personenbezogene Daten sind Name, Vorname, Privatanschrift, E-Mail-Adresse mit Namen, Standortdaten sowie auch Informaionen zu persönlichen Interessen oder dem Konsumverhalten. Als nicht personenbezogene Daten werden beispielsweise Handelsregisternummer, anonymisierte E-Mail-Adressen und generell anonymisierte Daten betrachtet. \cite{PersBezDaten_2021}

\subsection{Besonders schützenswerte personenbezogene Daten} \label{DSGVO_besonders}
Zu erwähnen ist, dass es besonders schützenswerte Daten einer natürlichen Person gibt. Die Verarbeitung dieser Daten ist entsprechend der DSGVO grundsätzlich untersagt. Zu diesen Daten gehören mitunter die rassische und ethnische Herkunft, politische Meinungen, religiöse oder weltanschauliche Überzeugungen, die Gewerkschaftszugehörigkeit, sowie genetische und biometrische Daten, Gesundheitsdaten oder Daten zum Sexualleben oder der sexuellen Orientierung einer natürlichen Person. Die im Artikel 9 Absatz 2 beschriebenen Fälle bilden die Ausnahme zur Verarbeitung dieser Daten. \cite{DSGVO_Art9}
% * Datenökonomie
\section{Datenökonomie} \label{datenoekonomie}
Unter Datenökonomie (engl. data economy) versteht man die wirtschaftliche Nutzung von Daten. Die wirtschaftliche Nutzung personenbezogener Daten wird dabei zusätzlich rechtlich durch die DSGVO eingeschränkt. Aber auch nicht personenbezogene Daten werden ökonomisch verwertet, so werden beispielsweise Prozess- und Maschinendaten gesammelt, aufbereitet und ökonomisch verwertet. \cite{bpb_2019} Meist werden die Daten dabei aggregiert und anderweitig aufbereitet, sodass sie im Falle der Erfassung von personenbezogenen Daten nicht mehr auf einzelne Personen zurückzuführen sind und daher fortan nicht mehr als personenbezogen betrachtet werden und entsprechend verwertet werden dürfen. \newline

\noindent Häufig werden Daten dazu verwendet, um Recherche- und Transaktionskosten zu senken und anhand derer präzise Entscheidungen zu treffen. Daher entsteht keine direkte Monetarisierung aus den Daten, weshalb der Wert dieser nicht eindeutig bestimmbar ist, sondern meist nur ein Schätzwert ist. Neben dieser Eigenschaft weisen Daten noch weitere Eigenschaften auf, die in der Datenökonomie eine wesentliche Rolle spielen. Einerseits haben Daten keinen Wertverlust durch deren Nutzung, was sie unerschöpflich macht. Andererseits kann deren Wert aber auch genau durch die Irrelevanz dieser sinken. Einzelne Daten haben meist wenig Relevanz und haben demzufolge einen geringen Wert. Erst durch die Aggregation mit anderen Daten entsteht ein Wert dieser, wodurch folglich Prognosen, Kennzahlen und weitere Vorgehen abgeleitet werden können. Daraus lässt sich gemäß eines Berichts der \gls{UN} eine Wertschöpfungskette für Daten herleiten. Diese besteht aus der Datenerfassung, der Speicherung und Organisation der Daten, der Analyse dieser und der folgenden Berichterstellung, woraus demnach die oben erwähnten Entscheidungen resultieren. \cite{un_2019} \newline

\noindent Aufgrund des enormen Potentials sind datengetriebene Geschäftsmodelle enstanden, die im Folgenden beleuchtet werden. Jedoch muss dabei vorerst unterschieden werden, wie differenziert die Datenerhebung stattfinden kann. Die International Data Corporation hat in einer Studie, zusammen mit der Open Evidence, drei Arten der Datenerhebung für die Europäische Kommission ermittelt. Zum einen sind dies \textit{First-Party-Daten}, die von Unternehmen über ihre Kunden gesammelt werden, um interne Analysen durchzuführen oder sie an Dritte zu verkaufen. \textit{Second-Party-Daten} dagegen sind Daten die in Kooperation mit anderen Unternehmen durch beispielsweise Werbeaktionen gesammelt werden und sich in gemeinsamen Besitz befinden. Bei der dritten Form der Datenerhebung, der sogenannten \textit{Third-Party-Daten}, werden Daten durch \gls{webScrapingG} oder durch den Kauf bei Datenanbietern erfasst. Dies ist der erste Abschnitt, der oben beschriebenen Wertschöpfungskette und damit auch die Grundlage des datengetriebenen Geschäftsmodells. Wurden die Daten erfasst, lässt sich ein Geschäftsmodell daraus ableiten. Oft sind datengetriebene Geschäftsmodelle \textit{\glqq as-a-service\grqq{}-Modelle}, das heißt die Daten dienen dem Nutzer als Service. So können zum Beispiel Produktvorschläge, Rabatte oder Werbung passend zum Kaufverhalten des Nutzers gemacht werden, die mitunter bei \gls{SaaS}- und \gls{PaaS}-Modellen zum Einsatz kommen. Ein weiteres Geschäftsmodell kann die Datenanalyse selbst sein, also \textit{Analytics-as-a-Service}, wobei ein Mehrwert für den Nutzer allein durch die aufbereiteten Daten entsteht. Als Beispiel wären hierbei Fitnesstracker zu nennen, die die Gesundheitsdaten der Nutzer erfassen und für diesen verständlich darstellen. Darüberhinaus kann ein weiteres Geschäftsmodell die Beratung von Unternehmen im Digitalisierungsprozess, \textit{Consultancy-as-a-Service}, sein, um diese in der eigenen Datenerfassung und -analyse zu unterstützen, damit diese wiederum First-Party-Daten auswerten können und ihr Geschäftsmodell verbessern können. \newline

\noindent Daraus leiten die Autoren des oben genannten Berichts drei Klassen von datengetriebenen Geschäftsmodellen ab. Dies sind zum einen Datennutzer, die ihre ermittelten Daten für die Verbesserung des eigenen Geschäftsmodells nutzen. Zum anderen sind dies Datenlieferanten, die analysierte Daten zur weiteren Nutzung bereitstellen und schließlich Datenvermittler, welche durch Beratung und Bereitstellung von Infrastrukturen hinsichtlich der Datenerfassung ein Geschäftsmodell betreiben. \cite{smart_2013} \newline

\noindent Zusätzlich werden zwei weitere Begriffe eingeführt: der \textit{Datenkonsument} und der \textit{Datenanbieter}. Dabei handelt es sich um Synonyme für das Unternehmen beziehungsweise den Nutzer. Das Unternehmen nimmt in der Datenökonomie vorwiegend die Rolle des Datenkonsumenten ein. Das heißt, es möchte anhand der gewonnen Daten oben genannte Mehrwerte generieren. Der Nutzer dagegen nimmt die Rolle des Datenanbieters ein, da dieser dem Datenkonsumenten die relevanten Daten anbietet und liefert. 

\chapter{Fallanalyse}
% Invisibly
\section{Invisibly} \label{invisibly}
Invisibly ist eine Plattform, bei der Benutzer ihre persönlichen Daten zur lizenzierten Freigabe bereitstellen können. Sie befindet sich derzeit noch in der Beta-Phase und ist bisher ausschließlich in Amerika verfügbar.

\subsection{Hintergrund und Ziel}
Die Plattform hatte ursprünglich das Ziel, bezahlungspflichtige Artikel leichter zugänglich zu machen. Benutzer der Webseite sollten Werbung ansehen und dafür Token erhalten, welche sie anschließend für den Zugang zu einzelnen Publikationen von Nachrichtenseiten und Wissenschaftsmagazinen ausgeben konnten -- ohne bei den einzelnen Diensten beispielsweise ein Abonnement abzuschließen. Da diese Idee nur wenige Personen überzeugte, wurde das Konzept überarbeitet. \cite{techRadarInvisibly_2021} \newline

\noindent Heute verfolgt Invisibly das Ziel, seine Benutzer mit einem fairen Anteil am Wert zu beteiligen, der bei der Verarbeitung persönlicher Daten entsteht. Die Plattform versucht dies aktuell auf zwei verschiedene Wege. Im Gegensatz zu anderen Plattformen, die Inhalte strategisch platzieren, versucht Invisibly als erstes seinen Benutzern auf Basis der freiwillig geteilten Daten hochwertige Inhalte zu präsentieren, die sie tatsächlich sehen wollen: \begin{quote}
    \textit{``We are creating an AI-powered platform with a feed that`s a true extension of you, rather than a feed strategically curated and targeted at you by the Big Tech and brands.'' \cite{invisiblyWhyPay_2021}}
\end{quote} Im zweiten Schritt zielt Invisibly darauf ab, seine Benutzer direkt am Verkauf der persönlichen Daten zu beteiligen, indem sie ihnen regelmäßig Geld auszahlt. \cite{invisiblyWhyPay_2021} Der Gründer von Invisibly, Jim McKelvey, beschreibt seine Plattform in einem Interview wiefolgt: \begin{quote}
    \textit{``We basically act as your agent. We try to sell you to advertisers and we give you the money. We ask how much you are willing to sell, package it up and sell it to the highest bidder.'' \cite{techRadarInvisibly_2021}}
\end{quote} Die Plattform verkauft die Rohdaten ihrer Nutzer jedoch nicht direkt -- stattdessen lizenziert Invisibly die Daten ihrer Nutzer. Auf diese Weise bleibt die individuelle Person Eigentümer der Daten und kann selbst bestimmen, welche Daten sie freigeben möchte. \cite{invisiblyGetPaid_2021} Anschließend vermittelt Invisibly die Lizenzen an Werbetreibende für Dienste und Produkte, die am besten zum jeweiligen Benutzer passen. Sie erhalten mit einer Lizenz solange Zugriff auf die persönlichen Daten, bis ein Benutzer seine Lizenz widerruft und somit die Freigabe beendet. Der Vorteil für Unternehmen besteht darin, dass Invisibly sie durch den Kauf von Lizenzen mit den Individuen zusammenbringt, die potentiell am besten zu deren Produkten passen. \cite{techRadarInvisibly_2021} Dabei ist es wichtig zu erwähnen, dass Benutzer von Invisibly ihre Daten freiwillig zur Lizenzierung freigeben und deshalb wahrscheinlich von einer Vermittlung an Dritte nicht abgeneigt sind. \newline

\noindent Im Folgenden wird erklärt, wie Individuen ihre persönlichen Daten bei Invisibly zu Geld machen können.

\subsection{Daten monetarisieren}
Benutzer sammeln bei Invisibly Punkte, wenn sie persönliche Daten mit der Plattform teilen. Gesammelte Punkte können später in US-Dollar umgerechnet und ausgezahlt werden. Derzeit gibt es drei verschiedene Möglichkeiten, Punkte zu sammeln: \newline

\noindent \textbf{Datenquellen verlinken:} Die meisten Punkte sammeln Benutzer, indem sie verschiedene Datenquellen in ihrem Invisibly-Profil hinterlegen. Für jede Datenquelle wird monatlich ein fester Betrag an Punkten gutgeschrieben, ähnlich wie bei einem passiven Einkommen. \cite{pymntsInvisibly_2021} Das Verknüpfen eines Bankkontos wird beispielsweise mit \textit{75 Punkten} pro Monat vergütet, wobei Werbepartner so Zugriff auf sämtliche Transaktionsdaten erhalten. Andererseits lassen sich verschiedene soziale Netzwerke bei Invisibly hinterlegen. Für jeden Account erhält ein Benutzer monatlich \textit{25 Punkte} und es werden aktuell die Plattformen Instagram, Twitter, TikTok, LinkedIn und Pinterest unterstützt. Zum Schluss bietet Invisibly eine Browser-Erweiterung an, welche den Verlauf der besuchten Webseiten aufzeichnet. Mit ihr sammeln Benutzer \textit{200 Punkte} pro Monat. \cite{instagramInvisibly_2021, lifewireInvisibly_2021} Abbildung \ref{fig:invisiblyProfile} zeigt im zweiten Screenshot den Tab \textit{Data Vault} der Invisibly-App, in welchem die hinterlegten Datenquellen verwaltet werden können. \newline

\begin{figure}[!ht]
	\centering
	\includegraphics[width=\textwidth]{invisibly_profile}
	\caption{Ansichten in der mobilen Invisibly-App: My Opportunities, Data Vault und Wallet \cite{behanceInvisibly_2021}}
	\label{fig:invisiblyProfile}
\end{figure}

\noindent \textbf{Fragen zur Person:} Benutzer können ihr Invisibly-Profil vervollständigen, indem sie verschiedene Fragen zu ihrer Person beantworten. So fragt die Webseite beispielsweise nach dem Geschlecht, dem höchsten Bildungsabschluss oder ob man Kinder hat. Jede Antwort wird dabei mit \textit{einem Punkt} vergütet. \cite{instagramInvisibly_2021} \newline

\noindent \textbf{Persönliches Feed:} Das persönliche Feed ist Invisibly's Vision von einer besseren Alternative zu Facebook und ähnlichen Beispielen. Das Feed enthält relevante Beiträge, die auf Basis der freiwillig geteilten, persönlichen Daten von verschiedenen Datenquellen ausgewählt und angezeigt werden. Diese Beiträge bestehen aus Bildern mit kurzen Texten und verweisen auf Diensteleistungen und Produkte von Werbetreibenden, wie beispielsweise Gegenstände zum Kaufen, Reise- und Ausflugsziele, Veranstaltungen, Gutscheine oder sonstige Angebote in Online-Shops und vieles mehr. Jeder Beitrag kann mit einem \textit{Like} oder \textit{Dislike} markiert werden -- auf diese Weise sammelt Invisibly Daten über persönliche Interessen und kann Vorschläge noch besser auf jede Person zuschneiden. Benutzer erhalten im Gegenzug für jeden Like oder Dislike \textit{einen Punkt}, wobei die maximale Anzahl an Punkten hier auf 20 Punkte pro Tag begrenzt ist. \cite{invisiblyWhyPay_2021} \newline

\noindent Das in Abbildung \ref{fig:invisiblyProfile} gezeigte erste Screenshot der Invisibly-App zeigt die Ansicht \textit{My Opportunities}, in welcher dem Benutzer noch nicht erledigte Möglichkeiten zum Sammeln von Punkten vorgeschlagen werden. An erster Stelle werden die Datenquellen aufgelistet, für welche noch kein Account hinterlegt ist, gefolgt von Fragen zur Person und dem persönlichen Feed. \newline

\noindent Die gesammelten Punkte können abschließend direkt in Geld umgewandelt werden: 100 Punkte entsprechen \$1 und eine Auszahlung ist ab 500 Punkten möglich. \cite{invisiblyWhyPay_2021} Eine Auszahlung ist über die \textit{Wallet} möglich, welche sowohl in der mobilen Version als auch über die in Abbildung \ref{fig:invisiblyWallet} dargestellte Webseite aufrufbar ist. Hier können Bentuzer neben ihrem aktuellen Punktestand auch einen Verlauf der gutgeschriebenen Punkte einsehen. Eine Auszahlung ist derzeit ausschließlich über den amerikanischen Zahlungsdienstleister PayPal möglich. \newline

\begin{figure}[!ht]
	\centering
	\includegraphics[width=0.9\textwidth]{invisibly_wallet}
	\caption{Wallet mit Kontostand und Verlauf der gesammelten Punkte \cite{behanceInvisibly_2021}}
	\label{fig:invisiblyWallet}
\end{figure}

\noindent Die Monetarisierung von persönlichen Daten bringt Invisibly's Nutzern derzeit ca. \$60 bis \$100 im Jahr, abhängig von der Anzahl verlinkter Datenquellen. Das liegt daran, dass aktuell ein flaches Bezahlmodell verwendet wird, bei dem jeder Nutzer gleich viel Geld bzw. Punkte für seine persönlichen Daten bekommt -- unabhängig davon, wieviel die Daten konkret für Werbetreibende wert sind. In Zukunft soll ein kompetitives Modell zum Einsatz kommen, bei dem Personen eine höhere Dividende erhalten, wenn die Daten für Werbetreibende mehr wert sind. \cite{pymntsInvisibly_2021} Der Gründer von Invisibly geht davon aus, dass ein schnell wachsender Marktplatz mit mehr Werbetreibenden und Individuen in wenigen Jahren bereits \$1000 pro Jahr für Nutzer des Dienstes erwirtschaften kann. \cite{techRadarInvisibly_2021} 


\chapter{Diskussion der Ergebnisse}

\chapter{Zusammenfassung}

% ==============================================
% Literaturverzeichnis
% ==============================================
\nocite{*}
\printbibliography
\addcontentsline{toc}{chapter}{Literaturverzeichnis}

% ==============================================
% Anhang
% ==============================================
\appendix
\chapter*{Anlagen} \label{Anlagen}
\addcontentsline{toc}{chapter}{Anlagen}
\setcounter{chapter}{1} 

\begin{figure}[!ht]
\centering
\includegraphics[width=\textwidth]{Bilddatei}
\caption{Bildbeschreibung}
\label{fig:Bildlabel}
\end{figure}

% Endblatt (weiß)
\clearpage
\thispagestyle{empty}
\begin{center}
\vfill
\phantom{© \makeatletter\@author\makeatother}
\end{center}

\end{document}
