% ------------------------------------------
% Glossar  
% keine Umlaute im entry verwenden   
\newglossaryentry{natuerlichePersonG}
{
name={natürliche Person},
description={\glqq Eine natürliche Person meint den Menschen als Rechtssubjekt und somit als Träger von Rechten und Pflichten.\grqq{} \cite{NatPerson_2018}}
}

\newglossaryentry{webScrapingG}
{
name={Web-Scraping},
description={\glqq Web-Scraping ist das Schürfen von Daten, also die Datenextraktion aus Webseiten.\grqq{} \cite{webScraping_2021}}
}

\newglossaryentry{brokerG}
{
name={Broker},
description={\glqq Der Broker wirkt als Makler bei reiner Geschäftsvermittlung auf fremden Namen und auf fremde Rechnung (Abschlussvermittlung) oder als Kommissionär in eigenem Namen und auf fremde Rechnung.\grqq{} \cite{broker_2018}}
}

\newglossaryentry{smartContractsG}
{
name={Smart Contract},
description={\glqq Smart Contracts sind digitale Verträge, die in einer Blockchain gespeichert sind und automatisch ausgeführt werden, wenn vordefinierte Bedingungen erfüllt sind\grqq{} \cite{smartContract_2022}}
}

\newglossaryentry{netzwerkeffektG}
{
name={Netzwerkeffekt},
description={\glqq Effekt, bei dem der Nutzen eines Gutes mit steigender Nutzerzahl (i.d.R.) zunimmt (positive Netzwerkeffekte).\grqq{} \cite{Netzwerkeffekt_2018}}
}

%============================================
% Akronyme
%============================================
\newglossaryentry{DSGVO}{
	type=\acronymtype, 
	name={DSGVO}, 
	description={Datenschutz-Grundverordnung}
}

\newglossaryentry{UN}{
	type=\acronymtype, 
	name={UN}, 
	description={United Nations}
}

\newglossaryentry{SaaS}{
	type=\acronymtype, 
	name={SaaS}, 
	description={Software as a Service}
}

\newglossaryentry{PaaS}{
	type=\acronymtype, 
	name={PaaS}, 
	description={Platform as a Service}
}

\newglossaryentry{PWA}{
	type=\acronymtype, 
	name={PWA}, 
	description={Progressive Web-App}
}