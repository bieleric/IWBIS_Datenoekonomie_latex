\section{Datum}
Datum ist ein Netzwerk, indem Nutzer ihre Daten dezentralisiert in einer Blockchain speichern können. Darüberhinaus ermöglicht es dem Nutzer diese Daten zu kaufen oder verkaufen und die Nutzung dieser entsprechend einzuschränken. Es ist also nicht nur ein Speicherort für Daten, sondern dient auch als Online-Datenmarktplatz und \gls{brokerG} zugleich.

\subsection{Ziel}

\subsection{Rollen im Datum-Netzwerk}
\textbf{Nutzer:} Nutzer sind diejenigen Personen im Netzwerk, die ihre persönlichen oder geschäftlichen Daten hinterlegen und zum Verkauf anbieten können. Sie nehmen dabei also die Rolle des Datenanbieters ein. Datum ermöglicht dem Nutzer zudem eine Einflussnahme auf die Privatsphäreeinstellungen, sodass folgende fünf Einstellungen getroffen werden können: 
\begin{enumerate}
	\item Das Teilen der Daten ist nicht erlaubt
	\item Das Teilen der Daten ist nur mit ganz bestimmten, identifizierten und dem Nutzer bekannten Datenkonsumenten gestattet
	\item Das Teilen der Daten ist wiederum nur mit ganz bestimmten, identifizierten und dem Nutzer bekannten Datenkonsumenten gestattet aber für eine Mindestgebühr
	\item Die Daten sind für jeden verfügbar
	\item Die Daten sind für jeden verfügbar aber für eine Mindestgebühr
\end{enumerate}

\noindent \textbf{Käufer:} Käufer können sich im Datum-Netzwerk den Zugriff auf die Daten der Nutzer erkaufen. Sie fungieren in diesem Netzwerk demzufolge als Datenkonsumenten, wobei sie jedoch nur einen eingeschränkten Zugriff auf die Daten haben, und zwar entsprechend der Nutzungsbedingungen des Nutzers. Auch sie können unterschiedliche Informationen offenlegen:
\begin{enumerate}
	\item Identität des Käufers
	\item Die allgemeine Datenschutzerklärung
	\item Die Zweckmäßigkeit
	\item Die Dauer der Aufbewahrung der Daten
	\item Das \textit{Datum-Network-Trust-Rating}
\end{enumerate}

\noindent \textbf{DAT-Token-Holder:} DAT-Token-Holder steuern das Netzwerk und ermöglichen Trankaktionen in dem Netzwerk. \newline

\noindent \textit{Hinweis: Auf weitere Rollen, wie beispielsweise Storage Nodes wird im Folgenden nicht eingegangen, da sich diese vorwiegend mit der technischen Komponente einer Blockchain befassen und im Sinne der Datenökonomie weniger relevant sind.}

\subsection{Daten im Datum-Netzwerk monetarisieren}

\subsection{Probleme}