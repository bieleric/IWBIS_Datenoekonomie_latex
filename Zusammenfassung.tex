Diese Arbeit behandelt unterschiedliche Möglichkeiten für Individuen, sich an der Datenökonomie zu beteiligen. Damit ist gemeint, dass Personen aktiv am Datenhandel mit entsprechendem Gegenwert teilnehmen und nicht -- wie bisher -- von Unternehmen als reine Datenquelle ausgenutzt werden. Hierfür wurden verschiedene Anbieter zur Monetarisierung von persönlichen Daten im Rahmen einer Fallstudie untersucht. \newline

\noindent Die Ergebnisse zeigen, dass Personen einerseits an verschiedenen \textit{Bonusprogrammen} teilnehmen oder sich andererseits auf einem \textit{Datenmarktplatz} registrieren können. Es ist zu beobachten, dass die Anforderungen der Benutzer bezüglich Kontrolle und Transparenz über die eigenen Daten von den Anbietern sehr unterschiedlich erfüllt werden. Die Datenmarktplätze befinden sich zudem in der Entstehungsphase und sind noch nicht am Markt etabliert. Es mangelt sowohl an Datenkonsumenten als auch Datenanbietern, weshalb der monetäre Gegenwert meist unattraktiv erscheint. Während Bonusprogramme deutlich etablierter sind, mangelt es bei ihnen vor allem an Kontrolle und Transparenz über die Daten sowie der Kenntnis darüber, dass Benutzer aktiv ihre persönlichen Daten monetarisieren. \newline

\noindent Neue Datenmarktplätze versuchen mit neuen Ideen, wie beispielsweise der Lizenzierung von Daten oder dem direkten Verkauf von Datensätzen, Personen aktiv an der Datenökonomie zu beteiligen. Es wird jedoch deutlich, dass aktuelle Probleme, wie etwa die fortlaufende Sammlung von Daten durch Dritte, damit nicht behoben werden können. Die untersuchten Fälle zeigen jedoch, dass Individuen bereits heute die Möglichkeit haben, ihre persönlichen Daten aktiv zu monetarisieren. Die stetig wachsende Verbreitung der Plattformen lässt zudem auf ein zunehmendes Bewusstsein für die Rolle von Individuen in der Datenökonomie schließen.