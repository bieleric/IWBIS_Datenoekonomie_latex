\section{Datenökonomie}
Unter Datenökonomie (engl. data economy) versteht man die wirtschaftliche Nutzung von Daten. Die wirtschaftliche Nutzung personenbezogener Daten wird dabei zusätzlich rechtlich durch die DSGVO eingeschränkt. Aber auch nicht personenbezogene Daten werden ökonomisch verwertet, so werden beispielsweise Prozess- und Maschinendaten gesammelt, aufbereitet und ökonomisch verwertet. \cite{bpb_2019} Meist werden die Daten dabei aggregiert und anderweitig aufbereitet, sodass sie im Falle der Erfassung von personenbezogenen Daten nicht mehr auf einzelne Personen zurückzuführen sind und daher fortan nicht mehr als personenbezogen betrachtet werden und entsprechend verwertet werden dürfen. \newline

\noindent Häufig werden Daten dazu verwendet, um Recherche- und Transaktionskosten zu senken und anhand derer präzise Entscheidungen zu treffen. Daher entsteht keine direkte Monetarisierung aus den Daten, weshalb der Wert dieser nicht bestimmbar ist. Neben dieser Eigenschaft weisen Daten noch weitere Eigenschaften auf, die in der Datenökonomie eine wesentliche Rolle spielen. Einerseits haben Daten keinen Wertverlust durch die Nutzung von Konsumenten, was sie unerschöpflich macht. Andererseits kann deren Wert aber auch genau durch die Irrelevanz dieser sinken. Einzelne Daten haben meist wenig Relevanz und haben demzufolge einen geringen Wert. Erst durch die Aggregation mit anderen Daten entsteht ein Wert dieser, wodurch folglich Prognosen, Kennzahlen und weitere Vorgehen abgeleitet werden können. Daraus lässt sich gemäß eines Berichts der \gls{UN} eine Wertschöpfungskette für Daten herleiten. Diese besteht aus der Datenerfassung, der Speicherung und Organisation der Daten, der Analse dieser und der folgenden Berichterstellung, woraus demnach die oben erwähnten Entscheidungen resultieren. \cite{un_2019} \newline

\noindent Aufgrund diesen Potentials sind auch datengetriebene Geschäftsmodelle enstanden, die im Folgenden beleuchtet werden. Jedoch muss dabei vorerst unterschieden werden, wie unterschiedlich die Datenerhebung stattfinden kann. Die International Data Corporation hat in einer Studie, zusammen mit der Open Evidence für die Europäische Kommission, drei Arten der Datenerhebung ermittelt. Zum einen sind dies \textit{First-Party-Daten}, die von Unternehmen über ihre Kunden gesammelt werden, um interne Analysen durchzuführen oder sie an Dritte zu verkaufen. \textit{Second-Party-Daten} dagegen sind Daten die in Kooperation mit anderen Unternehmen durch beispielsweise Werbeaktionen gesammelt werden und sich in gemeinsamen Besitz befinden. Bei der dritten Form der Datenerhebung, der sogenannten \textit{Third-Party-Daten}, werden Daten durch \gls{webScrapingG} oder durch den Kauf bei Datenanbietern erfasst. Dies ist der erste Abschnitt, der oben beschriebenen Wertschöpfungskette und damit auch die Grundlage des datengetriebenen Geschäftsmodells. Wurden die Daten nun erfasst, lässt sich ein Geschäftsmodell daraus generieren. Oft sind datengetriebene Geschäftsmodelle \textit{\glqq as-a-service\grqq{}-Modelle}, das heißt die Daten dienen dem Konsumenten als Service. So können zum Beispiel Produktvorschläge, Rabatte oder Werbung passend zum Kaufverhalten des Konsumenten gemacht werden, die mitunter bei \gls{SaaS} und \gls{PaaS} zum Einsatz kommen. Ein weiteres Geschäftsmodell kann die Datenanalyse selbst sein, also \textit{Analytics-as-a-Service}, bei ein Mehrwert bei dem Konsumenten allein durch die aufbereiteten Daten entsteht. Als Beispiel wären hierbei Fitnesstracker zu nennen, die die Gesundheitsdaten der Konsumenten erfassen und für diesen verständlich darstellen. Darüberhinaus kann ein weiteres Geschäftsmodell die Beratung von Unternehmen im Digitalisierungsprozess, \textit{Consultancy-as-a-Service}, sein, um diese in der eigenen Datenerfassung und -analyse zu unterstützen, damit diese wiederum First-Party-Daten auswerten können und ihr Geschäftsmodell verbessern können. \newline

\noindent Daraus leiten die Autoren des oben genannten Berichts drei Klassen von datengetriebenen Geschäftsmodellen ab. Dies sind zum einen Datennutzer, die ihre ermittelten Daten für die Verbesserung des eigenen Geschäftsmodells nutzen. Zum anderen sind dies Datenlieferanten, die analysierte Daten zur weiteren Nutzung bereitstellen und schließlich Datenvermittler, welche durch Beratung und Bereitstellung von Infrastrukturen hinsichtlich der Datenerfassung ein Geschäftsmodell betreiben. \cite{smart_2013} \newline