\section{Payback} \label{Payback}
Payback ist ein Bonussystem, welches 2000 in Deutschland eingeführt wurde und seit 2018 auch in Italien, Indien, Polen, Mexiko und Österreich verfügbar ist. Außerdem gehört es seit 2010 zur American Express Gruppe. \cite{Payback_Info}

\subsection{Rollen und Ziele}
Payback ist ein Treuepunkteprogramm, mit dem Kunden bei jedem Einkauf prozentual Punkte sammeln und diese gegen Prämien und Rabatte einlösen. Der Anbieter hat hierfür Verträge mit über 30 stationären Partnern und 600 Online Shops. \cite{Payback} Demzufolge existieren drei Rollen im Payback-Netzwerk: \textit{Payback} als Anbieter, der \textit{Kunde} und die \textit{Partnerunternehmen}. \newline

\noindent \textbf{Payback:} Das Ziel von Payback besteht vor allem darin, als zentrale Instanz Daten über seine Nutzer zu sammeln und diese auszuwerten. Auf diese Art kann die Plattform das Einkaufsverhalten über sämtliche Händler hinweg analysieren und seinen Nutzern dahingehend personalisierte Werbung anzeigen. \newline

\noindent \textbf{Partnerunternehmen:} Die Partnerunternehmen schließen mit Payback zur Teilnahme am Bonusprogramm einen Vetrag. Damit erhoffen sich die Händler eine stärkere Kundenbindung, da Besitzer einer Payback-Karte sich daran orientieren, ob sie beim Einkauf Payback-Punkte sammeln können. Darüber hinaus können Partnerunternehmen mithilfe von Payback-Coupons Werbeaktionen unterstützen, um einerseits Aufmerksamkeit auf Warengruppen und Termine zu richten und andererseits neue Kunden zu gewinnen, die bereits eine Payback-Karte besitzen. Letztendlich erhalten die Partner Zugriff auf Analyseergebnisse von Payback zum allgemeinen Kundenverhalten -- diese beinhalten nicht nur eigene Kunden, sondern alle Kunden der Payback-Partner. \newline

\noindent \textbf{Kunde:} Personen beantragen kostenlos eine Payback-Karte bzw. registrieren sich in der Payback-App, um am Bonusprogramm teilzunehmen. Diese zeigt der Kunde beim Bezahlvorgang an der Kasse vor und erhält einen nach Unternehmen unterschiedlichen Bonus in Form von Punkten gutgeschrieben. Online ist das Punktessammeln im Buchungsprozess bei einigen Händlers integriert, bei anderen muss der Kunde seinen Einkauf direkt über die Payback-Webseite oder App starten. Gesammelte Punkte können später für Rabatte, Gutscheine und weitere Prämien eingelöst oder direkt ausgezahlt werden.

\subsection{Datenerhebung}
Um am Payback-Programm teilzunehmen, müssen Benutzer folgende Basisdaten an Payback als Anbieter übermitteln: Name, Geburtsdatum und Adresse. Dabei ist es wichtig zu erwähnen, dass eine Teilnahme auch ohne Angabe dieser Daten möglich ist -- jedoch können dann keine gesammelten Punkte eingelöst werden. \cite{Payback_Teilnahme} Darüber hinaus können Personen freiwillig weitere Basisdaten angeben. \cite{Payback_Datenschutz} \newline

\noindent Bei der Datenerhebung gibt es jedoch nach der Art der Registrierung wesentliche Unterschiede. Beantragt der Kunde seine physische Payback-Karte bei einem der stationären Partner, so erhält auch dieser Zugriff auf die Basisdaten. Beantragt der Kunde hingegen eine neutrale Payack-Karte bei Payback selbst oder registriert sich online zur Verwendung der App, so bleiben alle persönlichen Basisdaten bei Payback. Partnerunternehmen übermitteln in jedem Fall die gesammelten Einkaufsdaten an die Plattform. Dies umfasst Payback-Kundennummer, eine Auflistung aller Waren/Dienstleistungen mit Preisen sowie den Rabattbetrag, Ort und Zeitpunkt des Einkaufs. Eine Ausnahme bilden Apotheken -- diese erhalten keinen Zugriff auf Basisdaten von Payback und melden auch keine Einkaufsdaten an Payback, da es sich gemäß Kapitel \ref{DSGVO_besonders} um besonders schützenswerte Daten handelt. \cite{Payback_Datenschutz} \newline

\noindent Payback speichert die Daten der Kunden nur solange diese aktiv am Programm teilnehmen. Steuerrechtlich werden diese jedoch zehn Jahre lang aufbewahrt. Die Kommunikationsdaten, zum Beispiel bei Kontaktierung des Service Centers werden nach spätestens sechs Jahren gelöscht. Darüber hinaus weist Payback ausdrücklich darauf hin, dass keine Adressdaten mit Partnerunternehmen geteilt werden. \cite{Payback_Datenschutz} \newline

\noindent Weiterhin kann der Nutzer frei entscheiden, ob seine Daten für Zwecke der Marktforschung und Werbung verwendet werden dürfen. Wenn der Nutzer einwilligt, werden passende werbliche Angebote für ihn ausgewählt. Diese erfolgen durch die Auswertung der Daten zur Mustererkennung im Einkaufsverhalten. Die Angebote werden entweder postalisch, per E-Mail oder per SMS mitgeteilt. Diese Daten werden durch Payback und die Partnerunternehmen zum Zwecke der Werbung verarbeitet. Der Kunde hat dabei jederzeit die Möglichkeit die Einwilligung zu widerrufen. Sobald die Einwilligung zur Verarbeitung der Daten für Werbezwecke vorliegt, wird anhand dessen ebenfalls eine Werbeplanung und Erfolgskontrolle durchgeführt. \cite{Payback_Datenschutz} \newline

\subsection{Daten bei Payback monetarisieren}
Eine Monetarisierung von Daten findet bei Payback für Individuen indirekt statt. Benutzer teilen ihre persönlichen Daten mit der Plattform und erhalten Punkte, die sie später für verschiedene Gegenwerte einlösen können. \newline

\noindent Die oben genannten Basisdaten müssen für die Teilnahme am Bonusprogramm zwingend mit Payback geteilt werden, obwohl Individuen dafür keine Punkte und somit keinen Gegenwert erhalten. Erst mit dem Teilen von Einkaufsdaten werden Personen mit Punkten vergütet. Der Betrag an Punkten variiert dabei von Unternehmen zu Unternehmen -- insgesamt werden aber Punkte im Wert von 0,5 bis 4\% der Kaufsumme gutgeschrieben. Häufig können Kunden mit Coupons oder Sonderaktionen jedoch einen vielfachen Betrag an Punkten sammeln. \newline

\noindent Gesammelte Punkte können anschließend für verschiedene Gegenwerte eingelöst werden. Einerseits ist eine direkte Auszahlung als Geldbetrag möglich. Ein Payback-Punkt hat dabei den Wert von einem Cent (0,01€) und eine Auszahlung ist ab 200 Punkten -- also 2€ -- möglich. \cite{Payback_Teilnahme} Andererseits können Kunden ihre Punkte auch für monetäre Gegenwerte einlösen: Einkaufs- und Geschenkgutscheine für viele Partner, Tanken und Shoppen bei Aral, Flugmeilen bei Lufthansa (1 Payback-Punkt entspricht 1 Meile), Spenden in der Payback-Spendenwelt oder direkt im Payback-Prämienshop für Produkte und Gutscheine. \cite{Payback_Einlösen} Abschließend ist es noch wichtig zu erwähnen, dass die gesammelten Punkte eines jeden Jahres zum 30. September verfallen, wenn diese nicht eingelöst werden. \cite{Payback_Teilnahme}
