\section{Ziel und Methodik}
Aus diesem Grund beschäftigen wir uns im Folgenden mit den verschiedenen Möglichkeiten für Personen, sich aktiv in der Datenökonomie zu beteiligen. In dieser Arbeit stellen wir dar, bei welchen Diensten individuelle Personen ihre Daten anderen Akteuren der Datenökonomie zur Verfügung stellen können und welchen Gegenwert sie dafür jeweils erhalten. Wir fokussieren uns dabei ausschließlich auf persönliche Daten, da diese sehr empfindliche Informationen über Individuen enthalten und deshalb für Unternehmen von großem Interesse sind. Zudem betrachten wir für jeden Fall, mit welcher Kontrolle und Transparenz die Daten geteilt werden. Unsere Forschungsfrage lautet deshalb:

\begin{center}
\textit{Welche Möglichkeiten gibt es für Personen, aktiv ihre persönlichen Daten ökonomisch zu verwerten?}
\end{center}

\noindent Um diese Frage zu beantworten beschäftigen wir uns zunächst mit den Begriffen \textit{personenbezogene Daten} und \textit{Datenökonmie}. Die Definition der Datenökonomie mit einer Abgrenzung zu anderen wirtschaftlichen Bereichen bildet die Voraussetzung für die darauffolgende Fallstudie. In dieser haben wir insgesamt sechs verschiedene Anbieter zur monetarisierung persönlicher Daten identifiziert. Obwohl die ausgewählten Unternehmen vielfältige Ziele verfolgen und sehr unterschiedlich funktionieren, lassen sie sich der Datenökonomie zuordnen. Nach Auswahl eines breiten Spektrums an Fallbeispielen analysieren wir jeden Dienst einzeln. Dabei gehen wir zunächst auf die Ziele der Anbieter ein und legen anschließend nachvollziehbar dar, wie Individuen der Datenökonomie ihre persönlichen Daten auf der jeweiligen Plattform monetarisieren können. Zusätzlich beschreiben wir in jedem Fall, welches Maß an Kontrolle und Transparenz den Benutzern offenbart wird. \newline

\noindent Im darauffolgenden Abschnitt werten wir die Ergebnisse der Fallstudie aus. Hierfür werden die Erkenntnisse zu den einzelnen Anbietern miteinander in Verbindung gebracht, um Gemeinsamkeiten und Unterschiede darzustellen. Im Mittelpunkt stehen dabei die Aspekte der Datenerhebung, Kontrolle und Transparenz über die Daten sowie der Gegenwert für Individuen. Dies ermöglicht uns abschließend, einige Themen -- wie beispielsweise die Ausnutzung von Personen -- kritisch zu hinterfragen und allgemeine Probleme abzuleiten. 