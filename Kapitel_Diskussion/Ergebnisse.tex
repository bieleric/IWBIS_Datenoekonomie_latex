\section{Ergebnisse}
Die in der Fallstudie betrachteten Beispiele zeigen, dass es verschiedene Möglichkeiten für Individuen gibt, ihre persönlichen Daten zu monetarisieren. Während sich die Plattformen in einigen Funktionen unterscheiden, funktionieren sie im Grunde nach demselben Prinzip. Als erstes werden Daten erhoben oder eingebunden, damit sie im zweiten Schritt aufbereitet und weiterverarbeitet werden können. Individuen erhalten hierfür im dritten Schritt einen Gegenwert. Im folgenden Abschnitt werden die Dienste anhand dieser Wertschöpfungskette in drei wesentlichen Punkten verglichen: der Art der Datenerhebung, der Kontrolle und Transparenz über erhobene Daten und abschließend dem Gegenwert, den Individuen für das Teilen ihrer persönlichen Daten erhalten.

\subsection{Datenerhebung}
Datenerhebung meint die Art, wie Dienste an persönliche Daten gelangen bzw. wie Individuen ihre Daten aktiv einbringen können. \newline

\noindent Bei den Bonusprogrammen von Payback und Kaufland muss zwischen den erhobenen Basisdaten für die Registrierung und den Einkaufsdaten unterschieden werden. Die Angabe der Basisdaten ist für die Teilnahme am jeweiligen Bonusprogramm unerlässlich. Über die Freigabe der Einkaufsdaten jedoch entscheidet der Nutzer durch das Vorzeigen der Payback- oder Kauflandkarte selbst. Die Registrierung des Einkaufs wird damit nicht im Bonusprogramm berücksichtigt und die Daten können demzufolge nicht monetarisiert werden. \newline

\noindent Die beiden Dienste BitsAboutMe und Invisibly unterscheiden sich in dem Punkt der Datenerhebung kaum. Auf beiden Plattformen können Benutzer verschiedene Datenquellen, wie beispielsweise Konten aus sozialen Medien oder Bankkonten, einbinden. Das Datum-Projekt verfolgt auch den Ansatz, dass Benutzer externe Dienste als Datenquellen verlinken und so ihre persönlichen Daten teilen. Zum aktuellen Zeitpunkt ist jedoch keine Liste mit möglichen Quellen bekannt. Bei dieser Art der Datenerhebung agieren Individuen lediglich insoweit aktiv, indem sie andere Plattformen verlinken, auf denen bereits personenbezogenen Daten aggregiert worden. \newline

\noindent Auf der anderen Seite steht die Datenerhebung durch das aktive Einbringen neuer persönlicher Daten. Bei BitsAboutMe und Invisibly haben Benutzer neben der Bekanntgabe von Registierungsdaten die Möglichkeit, eigene Aussagen zur Vervollständigung des Profils zu machen. Personen können Daten aus erster Hand aktiv einbringen, indem sie den eigenen Steckbrief ausfüllen oder mit dem persönlichen Feed interagieren. BitsAboutMe bietet zudem den Dienst an, Kassenzettel einzuscannen. 

\subsection{Kontrolle und Transparenz}
Kontrolle bezeichnet den Umfang der aktiven Teilnahme für Individuen beim Benutzen der Plattformen. Dabei ist es wichtig zu unterscheiden, wie groß die Kontrolle einerseits beim Teilen der Daten und andererseits im Nachhinein beim Entfernen ist. Transparenz meint im Kontext der Datenökonomie, in welchem Umfang Benutzer zu einem bestimmten Zeitpunkt über die Verwendung ihrer persönlichen Daten informiert werden. \newline

\noindent Wie bereits erwähnt, kann der Nutzer bei Payback und Kaufland selbst bestimmen, welche Einkäufe erfasst werden. Dieser kann bei der Erfassung eines Einkaufs jedoch nicht präzisieren, in welchem Umfang der Datensatz übermittelt wird. Beispielweise ist ein Ausschluss des Ortes, der Zeit oder einzelner Artikel nicht möglich. \newline

\noindent Bei den Diensten BitsAboutMe, Datum und Invisibly gilt, dass Benutzer bei der Datenerhebung im Allgemeinen die volle Kontrolle über ihre persönlichen Daten haben. Auf jeder der drei Plattformen ist es Individuen freigestellt, welche externen Datenquellen sie einbinden möchten und welche nicht -- keine Quelle wird erzwungen. Dabei ist es jedoch wichtig zu erwähnen, dass bei der Registrierung auf allen Plattformen die Angabe von Name und E-Mail-Adresse erforderlich ist. Betrachtet man die detailierte Kontrolle beim Einbinden einzelner Datenquellen, so ist festzustellen, dass Personen keine Entscheidungsgewalt haben. Das bedeutet, Benutzer können plattformseitig zunächst nicht einstellen, welche Datensätze konkret aus einer externen Datenquelle übernommen werden sollen. Dies liegt zum Teil daran, dass es sich an dieser Stelle meist um Rohdaten handelt, die erst später zu aussagekräftigen persönlichen Daten aufbereitet werden. \newline

\noindent Bezüglich der Kontrolle und Transparenz über persönliche Daten, die bereits mit den Plattformen geteilt wurden, unterscheiden sich die Fälle stark. \newline

\noindent Die Bonusprogramme Payback und Kaufland ermöglichen dem Nutzer keine nachträgliche Kontrolle der persönlichen Einkaufsdaten. Das heißt, der Nutzer hat nicht die Möglichkeit Datensätze nachträglich zu bearbeiten oder zu löschen. Eine Löschung ist nur insgesamt durch das Löschen des Kundenkontos möglich. Man kann bei Payback allerdings der Verarbeitung der Daten für die Marktforschung und zu Werbezwecken explizit widersprechen. Bereits verarbeitete Daten können hingegen nicht widerrufen werden. Über den Umfang der Verarbeitung der Einkaufsdaten und die Analyseergebnisse hat der Nutzer nach dem Abschluss eines Einkaufs keine Einsicht. Es wird lediglich in den jeweiligen Datenschutzbestimmungen darüber unterrichtet, wie und ob eine Weitergabe der Daten an Dritte erfolgt und über welchen Zeitraum die Daten gespeichert werden. \newline

\noindent BitsAboutMe bietet seinen Benutzern eine umfangreiche Kontrolle über die eigenen Daten an. Personen können nachträglich für jede eingebunde Datenquelle entscheiden, welchen konkreten Datensatz sie teilen bzw. löschen wollen. So lassen sich beispielsweise einzelne Beiträge aus dem Instragram-Feed oder abgespielte Lieder von Spotify entfernen. Zudem lassen sich ganze Datenquellen einfach entfernen, wodurch alte Daten automatisch von der Plattform gelöscht werden. Darüber hinaus werden Benutzer stets transparent über die Verwendung ihrer persönlichen Daten aufgeklärt. Bei jedem Angebot teilt die Plattform mit, sowohl welche Profil- als auch Rohdaten mit Dritten geteilt werden und zu welchem Zweck die Daten verwendet werden. Außerdem werden Individuen darüber informiert, wie die Verarbeitung abläuft. Dies beinhaltet insbesondere die Art der Speicherung -- beispielsweise Auswertung auf Servern des Datenkonsumenten -- und die Dauer des Zugriffs. \newline

\noindent Bei Datum stehen dem Nutzer insgesamt fünf Optionen zur Kontrolle über persönliche Daten zur Verfügung. Mit diesen können Personen einschränken, mit welchen Datenkonsumenten die Daten geteilt werden und ob generell eine Gebühr erforderlich ist. Es ist jedoch nicht ersichtlich, ob diese Einstellungen neben einer ganzen Datenquelle auch auf einzelne Datensätze anwendbar sind. Zudem bleibt die Frage offen, inwiefern Individuen ihre persönlichen Daten aus dem verteilten Speicher entfernen können. Mit jedem Datenzugriff durch Dritte werden Personen in einem Transparenzbericht über die Nutzung ihrer Daten informiert. Dabei ist es wichtig zu erwähnen, dass dieser Bericht erst \textit{nach} der Verwendung erstellt wird und somit bereits ein Verkauf stattgefunden hat. \newline

\noindent Obwohl Invisibly mit viel Kontrolle über die eigenen Daten wirbt, stehen Benutzern in der aktuellen Beta-Phase noch wenige Optionen zur Verfügung. Dabei ist es jedoch wichtig zu erwähnen, dass die Plattform nicht direkt getestet werden konnte und alle Informationen somit auf externen Quellen beruhen. Benutzer können geteilte Datenquellen jederzeit entfernen und dadurch den Zugriff für Invisibly widerrufen. Es bleibt aber die Frage offen, was danach mit den bereits durch Invisibly aggregierten Informationen geschieht: Da die Plattform persönliche Daten zur Erstellung eines detaillierten Profils weiterverarbeitet und keine Rohdaten direkt verkauft, können bereits verarbeitete Informationen nicht ohne Weiteres aus dem Profil entfernt werden. Individuen haben allerdings die Möglichkeit, Invisibly direkt zu kontaktieren und um die Löschung von persönlichen Daten zu bitten. Auf diese Weise können sie ebenfalls einen Download ihrer Daten anfordern, welcher als Transparenzbericht über das eigene Profil dient.

\subsection{Gegenwert für Individuen}
Auch wenn sich die untersuchten Dienste teilweise deutlich in ihren Funktionen unterscheiden, erhalten Benutzer in jedem Fall einen Gegenwert für ihre aktive Teilnahme an der Datenökonomie. \newline

\noindent Bei den meisten Plattformen handelt es sich dabei um einen monetären Gegenwert. Während Individuen bei BitsAboutMe für jede Transaktion direkt Geld auf ihr Benutzerkonto erhalten, sammeln sie bei Invisibly zunächst Punkte, die sie letztendlich auch in Geld umrechnen und auszahlen können. Bei Payback werden ebenfalls Punkte gesammelt. Diese können sich Benutzer entweder direkt in Form von Geld auf das eigene Bankkonto überweisen lassen, in Gutscheine umwandeln oder im Prämien-Shop zum Kauf von Gütern verwenden. Ähnliches gilt für die Kaufland-Card: Hier erhalten Benutzer zunächst geldwerte Rabatte und gesammelte Treuepunkten können später für Coupons eingelöst werden. Auf der Datum-Plattform hingegen sammeln Benutzer sogenannte DAT-Token, welche auf Kryptowährungs-Börsen verkauft werden können.\newline

\noindent Dem monetären Erlös stehen alternative Gegenwerte gegenüber, welche auf den einzelnen Plattformen stark variieren. Auf BitsAboutMe erhalten Personen ausführliche Informationen zu ihren persönlichen Daten. Hierfür analysiert der Dienst sämtliche Datensätze aus geteilten Datenquellen und liefert Zusammenfassungen, beispielsweise zum Nutzerverhalten nach der Zeit. Informaionen bekommen Individuen auch auf Invisibly als Gegenwert. Bei dieser Plattform handelt es sich dabei um das persönliche Feed, welches Nutzern für sie relevante Beiträge anzeigt -- basierend auf umfangreichen Analysen der persönlichen Daten. Bei Payback und Kaufland werden persönliche Daten verwendet, um Personen individuelle Werbung anzuzeigen und sie somit auf relevante Produkte aufmerksam zu machen.