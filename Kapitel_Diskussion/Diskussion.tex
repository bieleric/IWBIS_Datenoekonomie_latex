\section{Diskussion der Ergebnisse}

- Validierung der pers. Daten für Kunsomenten: 
    *BitsAboutMe Abgleich Konto vs Kassenzettel
    *Datum Trust-Ranking-System gut, aber noch offen
- Kontrolle/Transparenz variiert stark von Dienst zu Dienst
    *Weiterverarbeitung
    *Entfernen von Daten
- BitsAboutMe und Invisibly sehr ähnlich aber trotzem unterschiedlich
- BAM und Invisibly VS Datum
- Verschiedene Motivation für Kunden:
    * BAM Analyseservices
    * Invisibly persönliches Feed, nicht manipulativ


- Generelle Probleme:
    * Plattformen können Missbrauch nicht verhindern, d.h. Konsumenten können Daten weiterverwenden oder verkaufen => Daten nicht rückrufbar
    * Andere Dienste können weiterhin Daten sammeln und verkaufen (Kontrolle auf Plattform vs. Kontrolle bei Datenquelle)
    * Scope: Händlergebunden vs. Breit Verfügbar

- Wert der Daten bzw. Wert für Kunden schlecht vergleichbar
    * Dienste in unterschiedlichen Stadien (Fertig, Beta, Konzepte)
    * teilweise nicht praktisch testbar
    * unterschiedliche Preismodelle für Datensätze (BAM dynamisch, Invisibly flat)

- Erkenntnis:
    * Personen beteiligen sich zwar, aber verkaufen nicht aktiv auf knopfdruck Daten
    * Eher Daten bereitgestellt und warten auf Angebote von Käufern
    * Dennoch aktive Teilnahme an der Datenökonomie