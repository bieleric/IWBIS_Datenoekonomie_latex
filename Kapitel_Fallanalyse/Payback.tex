\section{Payback} \label{Payback}
Payback GmbH ist die Tochtergesellschaft der Lovalty Partner GmbH, welche wiederum ein Teil der American Express Gruppe ist. \cite{Payback_Info} Payback ist allgegenwärtig in Deutschalnd bekannt, entweder per Kundenkarte oder per App, mit welcher sich Punkte sammeln lassen. \newline
\noindent Payback wurde im Jahre 2000 gegründet und existiert in Deutschland, Italien, Indien, Polen, Mexiko und Österreich. \cite{Payback_Info} \newline

\subsection{Ziel}
In vielen teilnehmenden Einzelhandelsketten wird an der Kassse beim bezahlen nach Payback gefragt. Payback ist dabei ein Bonusprogramm mit dem Kunden bei jedem Einkauf Punkte sammeln können. Dies ist ganz einfach zu erklären, Payback hat über 600 Vertragspartner offline wie online. Dabei kann der Kunde entweder seine Payback Karte vorlegen und diese einscannen lassen oder über das Smartphone die App vorzeigen, um Punkte zu sammeln. \cite{Payback} \newline

\noindent Das Ziel des Karten- bzw. Appinhabers ist dabei viele Punkte zu sammeln, um diese in Coupons und Prämien umtauschen zu können. Womit er für andere Produkte am Ende weniger bezahlen muss. \newline

\noindent Das Ziel von Payback ist es Daten zu generieren, um diese auswerten zu können, z. B. um das Kaufverhalten zu analysieren und dem Kunden dahingehend personalisierte Werbung zu zeigen. Ein weiteres Ziel ist es Verbraucher zu animieren mehr Geld auszugeben, als sie es ohne Rabatte oder Coupons getan hätten. \cite{web}


\subsection{Datenfreigabe und -verarbeitung}
Möchte man am Bonusprogramm von Payback teilnehmen, so benötigt Payback persönliche Daten. Die persönlichen Daten beschränken sich auf den Namen, das Geburtsdatum und die Anschrift. Diese Daten werden als Basisdaten betitelt. Ohne diese Daten können zwar Punkte gesammelt werden, allerdings können diese nicht eingelöst werden. Weiterhin besteht die Möglichkeit freiwillig mehr als die Basisdaten anzugeben. Für das Payback Programm verarbeitet Payback die Basisdaten und die freiwilligen Angaben für die Anmeldung und die Abwicklung. Nebst werden die Richtigkeit und die Vollständigkeit der Adressdaten überprüft damit die Post den Kunden erreicht. Die Daten und ggf. Änderungen der Daten werden auch an die Partnerunternehmen übermittelt, dies gilt nur für das Unternehmen woher der Kunde die Kundenkarte erhalten hat. Eine Ausnahme bildet die Apotheke, dahin werden die Daten nicht übermittelt. Für weitere Übermittlungen an Dritte oder Partnerunternehmen finden nur statt, wenn diesen gesondert vom Kunden zugestimmt wurden. \newline

\noindent Sobald der Kunde einen Einkauf tätigt, meldet das Unternehmen die Kundennummer, Rabattdaten an Payback.  Ausnahme bildet hier wieder die Apotheke. \newline

\noindent Eine weitere Möglichkeit Punkte zu sammeln bilden Coupons. Diese werden bereitgestellt von den Partnerunternehmen und bei Aktivierung der Coupons durch den Kunden, informiert Payback das jeweilige Unternehmen darüber, damit diese bei den Einkäufen berücksichtigt werden können. \newline

\noindent Wenn ein Kunde die Punkte einlösen möchte, kann der Kunde bestimmen wie viele Punkte er einlösen möchte. Diese übermittelt das Unternehmen an Payback zur Abrechnung. \newline

\noindent Weiterhin werden Kommunikationsdaten übermittelt, dies sind alle Angaben, die über das Kundenterminal oder über das Service Center, zur Bearbeitung eines Anliegens des Kunden anfallen. \newline

\noindent Der Kunde kann entscheiden, ob seine Daten für Zwecke der Marktforschung und Werdbung verwendet werden dürfen. Wenn der Kunde einwilligt werden passende werbliche Angebote für Ihn ausgewählt. Diese erfolgen durch die Auswertung der Daten zur Mustererkennung im Einlaufsverhalten. Die Angebote werden entweder postalisch oder per eMail oder per SMS mitgeteilt. Diese Daten werden an Payback und die Partnerunternehmen zum Zwecke für Werbung verarbeitet. Der Kunde hat dabei jederzeit die Möglichkeit die Einwilligung zu widerrufen. Sobald die Einwilligung zur Verarbeitung der Daten für Werbezwecke vorliegt wird anhand dessen ebenfalls eine Werbeplanung und Erfolgskontrolle durchgeführt. \newline

\noindent Payback speichert die Daten der Kunden, solange diese aktiv amm Programm teilnehmen, anschließend werden diese gelöscht. Steuerrechtlich werden diese jedoch 10 Jahre lang aufbewahrt. Die Kommunikationsdaten, zum Beispiel bei Kontaktierung des Service Centers werden nach sptestens 6 Jahren gelöscht.
\cite{Payback_Datenschutz} \newline

\noindent Payback bietet 3 Kartenmodelle für das Punktesammelsystem an. Zum einen die PAyback Karte, welche dauerhaft kostenlos ist, sowie das der Kunde bei allen VErtragspartnern Punke sammeln können. Das zweite Modell ist die Payback American Express Karte, welche ebenfalls kostenlos ist. Hierbei kann der Kunde bei jeder Bezahlung mit dieser Karte Punkte sammeln sei es bei der BEazhlung von Einkäufen bei Partnerunternehmen sowie nicht Partnerunternehmen. Der Vorteil ist bei dieser Karte, dass die Punkte nicht verfallen können und es gibt sogar extra 5000 Payback Punkte bei Kartenabschluss. Die Payback Visa Karte ist das dritte Modell, diese besitzt ebenfalls kein Punkteverfall. Weitere Vorteile dieser Karte sind 0€ Jahresgebühr, 0€ Gebühren bei Bargeldabhebung und auch außerhalb von Partnerunternehmen kann der Kunde bei BEzahlung mit der Karte Punkte generieren. Positiv sei zu erwähnen das Payback einen TÜV geprüften Datenschutz besitzt. \cite{Payback_Karten} \newline

\subsection{Daten auf Payback monetarisieren}