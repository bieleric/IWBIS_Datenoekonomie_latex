\section{Motivation}

Die Verbreitung des Internets auf Computern und mobilen Endgeräten hat zu einer Produktion riesiger Datenmengen geführt. Diese Daten -- insbesondere personenbezogene Daten -- haben sich zu einer wichtigen Ressource entwickelt, mit der Unternehmen einen Mehrwert erzeugen. Im Weiteren entstand daraus eine Datenökonomie, die allein auf dem Handel und der Verarbeitung personenbezogener Daten beruhen. \cite{humanDemand_2020} Dabei werden Daten meist jedoch nicht direkt von den Erzeugern, den individuellen Personen, verkauft. Stattdessen erheben große Internetdienste Anspruch auf die Daten ihrer Nutzer durch Klauseln in den Nutzungsbedingungen und verkaufen diese dann an Dritte weiter -- oftmals ohne Nutzer im Einzelfall darüber in Kenntnis zu setzen. \cite{monetizingData_2016}
\\
\\
Obwohl Individuen durch Erzeugung von Daten die Grundlage solcher Ökosysteme bilden, werden sie von Unternehmen bislang nur als Datenquelle gesehen und somit ausgenutzt. Viele Personen werden sich in letzer Zeit allerdings zunehmend bewusst, dass Technologieunternehmen diese Daten sammeln und verkaufen. Darüber hinaus wird ihnen deshalb bewusst, dass ihre persönlichen Daten einen gewissen Wert haben und dass sie mehr Kontrolle über deren Verwendung haben sollten. Rantanen und Koskinen fanden in einer Studie heraus, dass Personen zwei wesentliche Forderungen an eine Faire Datenökonomie stellen: transparente Kommunikation und eine aktive Rolle im Handel mit den eigenen Daten. Eine aktive Rolle in der Datenökonomie meint dabei, dass Individuen die Nutzung ihrer eigenen Daten kontrollieren und für die Teilnahme einen entsprechenden Gegenwert erhalten wollen. \cite{humanDemand_2020}
\\
\\
Bereits im Jahr 2016 stellte Batineh et. al jedoch fest, dass es an Plattformen mangelt, auf denen Individuen als primäre Datenerzeuger ihre Daten für einen entsprechenden Gegenwert selbst anbieten können. \cite{monetizingData_2016} Aus diesem Grund beschäftigen wir uns im Folgenden mit der Frage:

\begin{center} 
Welche Möglichkeiten gibt es für Personen, aktiv ihre Daten ökonomisch zu verwerten? 
\end{center}